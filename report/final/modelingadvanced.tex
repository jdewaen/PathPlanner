\section{Improving the MILP problem}
\label{section:modelingadvanced}


Vehicles have a maximum velocity they can achieve. Calculating the velocity of the vehicle means applying Pythagoras' theorem on the axis of the velocity vector. This is not possible using only linear equations. However, it can be approximated to an arbitrary degree using multiple linear constraints. The components of the velocity vector can be positive or negative, but only the absolute value matters for the actual velocity. 
\begin{figure}[h]
ABS \\
MAXSPEED
\end{figure}


Because obstacles make the problem non-convex and thus require integer constraints to model, the execution time scales very poorly with the amount of obstacles. This can be mitigated by only modeling a certain amount of obstacles relatively close to the vehicle, while limiting the vehicle to a convex region which does not overlap any of the ignored obstacles. Modeling this convex allowed region is very similar to modeling the obstacles, except that this time no integer variables are needed.

\begin{figure}[h]
ACTIVE REGION\\
\end{figure}


\begin{figure}[h]
\begin{math}
minimize \quad N - \mathlarger{\sum}_{t=0}^{t \leq N} fin_t \\
fin_0 = 1 \\ 
fin_{t+1} = fin_t \vee cfin_{t+1},  \quad 0 \leq t < N \\ \\
cfin_{pos,t} =  \mathlarger{\mathlarger{\bigwedge_{i = 0}^{i < Dim(\boldsymbol{pos}_t)}}} |pos_{t,i} - pos_{goal, i}| < pos_{tol},  \quad 0 \leq t \leq N \\ \\ \\
cfin_{vel,t} = 
\begin{cases*}
\mathlarger{\bigwedge_{i = 0}^{i < Dim(\boldsymbol{vel}_t)}} |vel_{t,i} -vel_{goal, i}| < vel_{tol}& if $\boldsymbol{vel}_{goal}$ exists, $0 \leq t \leq N$  \\
true & otherwise, $0 \leq t \leq N$ 
\end{cases*} \\ \\ \\
cfin_t =  cfin_{pos,t} \bigwedge cfin_{vel,t} \quad 0 \leq t \leq N
\end{math}
\end{figure}

