\section{Segmentation of the MILP problem}
\label{section:segment}

\subsection{Introduction}
TODO:REWRITE
Key to making MILP scale is dividing the problem into smaller pieces. Goal cannot be reached in a single segment, so needs to be guided some other way. Need to consider that multiple corners in single segment is problematic, so segments should minimize the amount of corners. Will use theta* and detect corners, use corners to construct segments.\\
Smaller segments allow lower upper bound on time needed, obstacles still an issue. Maximum possible distance vehicle can travel is first approximation, but still not enough in dense cities. Need to significantly cut down on amount of obstacles safely. Will demonstrate that not every obstacle is equally important, how to find the important obstacles and safely compress other obstacles using genetic algorithm.\\
Stress that the solution chosen can be swapped out. What's written here is one option.




%Short paths can be solved directly, but longer paths with a large amount of obstacles nearby take an extremely long time to solve. A solution to this problem is to divide the path into smaller segments. Solving the smaller segments is proportionally disproportionately easier than solving the full path, but at the cost of optimality and stability.\\
\subsection{Guiding the segmented path}
The first issue is that because the goal cannot be reached immediately, the goal function needs to change.  \\
One option is to simply get as close as possible to the goal during each segment. Because the distance that can be traveled during a segment is limited, the amount of obstacles that need to be modeled is also limited. This works well when the world is very open with little obstacles. However, this greedy approach is prone to getting stuck in dead ends in more dense worlds like cities.
\\
The second option is finding a complete path using a method that's easier to compute. An algorithm like A* can be used to find a rough estimation for the path. This A* path is the shortest path, but does not take constraints or the characteristics of the vehicle into account. A very curvy direct path may be the shortest, but a detour which is mostly straight and allows for higher speeds may actually be faster. 
\\
It is possible to use more advanced algorithms that model more of the constraints. This will significantly improve the speed and quality of the constraint optimization step, but it will also come at a performance cost. A balance needs to be found between the preprocessing step and the constraint optimization step. The preprocessing step needs to do just enough so the optimization step can be solved in an acceptable amount of time. 
\\
\subsubsection{Thetha* implementation}
\begin{figure}
A* 4 vs A* 8 vs Theta* comparison
\label{figure:thetastarcompare}
\end{figure}




I have decided to use Theta*. This is a variant  of A* that allows for paths at arbitrary angles instead of multiples of 45 degrees. The main reason for this is that it eliminates the ``zig zags'' that A* produces. This makes the next step of the preprocessing much easier.\\
\subsubsection{Scalability of Theta*}
TODO: REWRITE
Theta* provides the information that makes it possible to make MILP fast, but Theta* in itself is intractable. Why is this not simply moving the issue?\\
Theta* is indeed intractable, but still much better than MILP. Holomomic vs non-holonomic. Concept of navigation mesh as part of map, often used in video games to allow AI to navigate large worlds. Also same concept behind PRM. Only needs to pre precomputed once per map. Can serve as heuristic for Theta*, or completely replace it.

\subsection{Detecting corner events}
With an initial path generated, the next problem is dividing it into segments. Dividing the path into equal parts presents problems, because when solving each segment, the solver has no knowledge of what will happen in the next segment. This is especially problematic when the vehicle needs to make a tight corner. If the last segment ends right before the corner, it may not be possible to avoid a collision.\\
In Euclidian geometry, the shortest path between two points is always a straight line. When polygonal obstacles are introduced between those points, the shortest path will be composed of straight lines with turns at one or more vertices of the obstacles. The obstacle that causes the turn will always be on the inside of the corner. This shows why corners are important for another reason: They make the search space non-convex. For obstacles on the outside of the corner it is possible to constrain the search space so it is still convex, but this is not possible for obstacles on the inside.\\
Because of these reasons, isolating the corners from the rest of the path is advantageous. With enough buffer before the corner, the vehicle is much more likely to be able to navigate the corner successfully. It also means that the computationally expensive parts of the path are as small as possible while still containing enough information for fast navigation through the corner.
\\
The reason for using Theta* becomes clear now. Every single node in the path generated by the algorithm is guaranteed to be either the start, goal or near a corner. A corner can have more than one node, so nodes which turn in the same direction and are close to each other are merged into a single corner event. A corner event is also generated for nodes which are alone.
\subsubsection{Tight Coupling with Theta*}
This algorithm relies on assumptions from Theta*, so if algorithm changes, this will need to be updated.
\subsection{Generating path segments}
These corner events are in turn grown outwards to cover the approach and departure from the corner. How much depends on the maximum acceleration of the vehicle. As a rule of thumb: If the vehicle can come to a complete stop from its maximum speed before the corner, it can also successfully navigate that corner. When corners appear in quick succession, their expanded regions may overlap. In that case, the middle between those corners is chosen. Long, straight sections are also divided into smaller path segments.\\
One of the main goals of segmenting the path is to reduce the amount of obstacles. This means that every segment has a set of obstacles associated with it, being the obstacles that need to be modeled in the optimization step. Not only the obstacle that ``causes'' the corner is important, but obstacles which are nearby are important as well. Obstacles on the outside of the corner also may play a role in how the vehicle approaches the corner. To find all potentially relevant obstacles, the convex hull of the (Theta*) path segment is calculated and scaled up slightly. Every obstacle which overlaps with this shape is considered an active obstacle for that path segment. The convex hull step ensures that all obstacles on the inside of the corner are included, while scaling it up will cover any restricting obstacle on the outside of the corner.
\subsection{Generating the active region for each segment}
Even though the most important obstacles are taken care of, all other obstacles also need to be represented. To do this, a convex polygon is grown around the path. This polygon may intersect with the active obstacles (since they will be represented separately), but may not intersect any other obstacle. The polygon is grown using a genetic algorithm. It uses a single mutator which nudges the vertices of the polygon while ensuring it stays convex and does not intersect itself or any non-active obstacle. The genetic algorithm is just one way to generate the convex polygon which represents the active region. Deits and Tedrake\cite{Deits2015} have demonstrated how another algorithm can solve the same problem.