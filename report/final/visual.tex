\section{Visualizing solution}
\label{section:visual}
Using MILP to solve the trajectory planning problem makes my algorithm flexible. Different constraints can be added and removed without having to change complex algorithm. The solver takes those changes into account and still finds a solution. This is a form of declarative programming. Instead of defining "how" to solve a problem (like with imperative programming) you define the problem itself.\\
This makes it much easier to experiment with different variants of the problem, however it does also have downsides. One of those downsides is that it becomes harder to understand why the solution to the problem is what it is. Especially since MILP solvers don't provide a lot of useful insight during execution.\\
This problem is made worse by the fact that the trajectory planning problem is an optimization problem. We're not just interested in having any solution, but instead we want a good or possibly even the best solution.\\
These factors make the MILP solvers a "black box". This is especially problematic when they fail to find a solution. Most solvers will inform you which constraint caused the failure, but that constraint is not necessarily the one that is wrong. Another problem is figuring out if all constraints are actually modeled properly. A badly modeled constraint may have no effect at all, or a different effect than intended. Gaining a deep understanding of what is happening is an issue as more and more constraints are added and the interactions between them increase.\\
For this reason, I spent a significant amount of time building a visualization tool which displays not only the solution, but also the constraints of the MILP problem and other debugging information. This proved to be a critical part of the development cycle of the algorithm.



black box solver means debugging is tricky. Need other way to get insight in what's happening. Visualization tool \\ \\

- obstacles \\
- pre path \\
- corners \\
- path segments\\
- active region\\
- active obstacles w/ slack vars\\
- solver path\\
- exact value readout \\