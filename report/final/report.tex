% *!TeX spellcheck = nl_NL
\documentclass[
master=cws,
masteroption=ai,
english,
%oneside
]{kulemt}
%	options include 12pt or 11pt or 10pt
%	classes include article, report, book, letter, thesis

\title{Scalable Multirotor UAV Trajectory Planning using Mixed-Integer Linear Programming}
\author{Jorik De Waen}
\date{\today}

\setup{
title={Scalable Multirotor UAV Trajectory Planning using Mixed-Integer Linear Programming},
promotor={Prof.\,dr.\,Tom Holvoet},
assessor={Dr.\,Mario Henrique Cruz Torres \and Dr.\,Bart Bogaerts},
assistant={Hoang Tung Dinh \and Dr.\,Mario Henrique Cruz Torres}
}

\setup{filingcard,
translatedtitle={Schaalbare Traject Planning voor Onbemande Multirotor Luchtvaartuigen met Gemengd Geheeltallig Lineair Programmeren},
shortabstract={
This thesis presents a new highly scalable offline trajectory planning algorithm for multirotor UAVs. The algorithm is based around Mixed-Integer Linear Programming (MILP). Previous approaches which used MILP for trajectory planning suffered from scalability limitations in large environments with many obstacles. The new approach can handle tens of thousands of polygonal obstacles successfully on a typical consumer computer. 
\endgraf
This performance is achieved by dividing the problem into many smaller MILP subproblems using two sets of heuristics. Each  subproblem models a small part of the trajectory. The subproblems are solved in sequence, gradually building the desired trajectory.
\endgraf
The first set of heuristics generate each subproblem in a way that minimizes its difficulty, while preserving stability. The second set of heuristics select a limited amount obstacles to be modeled in each subproblem, while preserving consistency.
\endgraf
To demonstrate that this approach can scale enough to be useful in real, complex environments, it has been tested on maps of two cities with trajectories spanning over several kilometers.
},
udc={004.8},
keywords={Trajectory Planning, UAV, Mixed-Integer Linear Programming, MILP, Constraint Pptimization},
articletitle={Scalable Multirotor UAV Trajectory Planning using Mixed-Integer Linear Programming}
}

\chapterstyle{ell}

\usepackage{amsmath}
\usepackage{relsize}
\usepackage{mathtools}
\usepackage{graphicx}
\usepackage{caption}
\usepackage{subcaption}
\usepackage{hyperref}
\usepackage{amssymb}
\usepackage{algpseudocode}
\usepackage{algorithm}
\usepackage{algorithmicx}
\usepackage[final]{pdfpages}

\usepackage{setspace}
%\doublespacing


\graphicspath {{img/}}

\algnewcommand\algorithmicforeach{\textbf{for each}}
\algdef{S}[FOR]{ForEach}[1]{\algorithmicforeach\ #1\ \algorithmicdo}
\newcommand{\Break}{\State \textbf{break} }
\renewcommand{\Return}{\State \textbf{return} }

\newtheorem{hyp}{Hypthesis}

\renewcommand{\topfraction}{.85}
\renewcommand{\bottomfraction}{.7}
\renewcommand{\textfraction}{.15}
\renewcommand{\floatpagefraction}{.66}
\renewcommand{\dbltopfraction}{.66}
\renewcommand{\dblfloatpagefraction}{.66}
\setcounter{topnumber}{9}
\setcounter{bottomnumber}{9}
\setcounter{totalnumber}{20}
\setcounter{dbltopnumber}{9}

%\setup{coverpageonly}

\begin{document}

\begin{preface}
This Master's Thesis marks the end of a year full of challenges, hard work and a good amount of fun. I couldn't have asked for a better subject.
\par
I also couldn't have asked for better supervision, cooperation and support. I would like to thank Prof. dr. Tom Holvoet for his support and constructive feedback throughout the project. I am also extremely grateful to Hoang Tung Dinh and Dr. Mario Henrique Cruz Torres for their excellent guidance and commitment. Without them, this Master's Thesis wouldn't have been possible.
\par
I'm also eternally indebted to my parents for their unwavering support. Without your help in these challenging times I would not have made it. I also want to thank Maarten for being so patient with me. I could always count on you. I owe you one. 
\par
Last but certainly not least, I have to thank Caroline. Your endless love and care never fail to turn a bad day into a good one. Thank you for always standing by my side.
\par

\end{preface}

\tableofcontents*

\begin{abstract}
This thesis presents a new highly scalable offline trajectory planning algorithm for multirotor UAVs. The algorithm is based around Mixed-Integer Linear Programming (MILP). Previous approaches which used MILP for trajectory planning suffered from scalability limitations in large environments with many obstacles. The new approach in this thesis is capable of generating a trajectory through 2D space in environments based on data from real cities, spanning several square kilometers with tens of thousands of polygonal obstacles.
\par
This improvement in performance is achieved by dividing the trajectory planning problem in many smaller subproblems. Each subproblem models a small part of the trajectory. To split the problem into subproblems, the algorithm finds a path to the goal first. Unlike a trajectory, a path is time-independent and does not take the UAV's dynamics into account. The subproblems are generated around the turns in this path, which limits the amount of obstacles that need to be modeled in each individual subproblem.
\par
The algorithm was tested on several different scenarios. Some of these scenarios take place in worlds based on maps of San Francisco and Leuven. The experiments show that the scalability of the algorithm is mainly limited by the density of the distribution of obstacles in the world. However, even with the density of obstacles in the Leuven data set being higher than necessary for this application, the algorithm can still reliably plan a trajectory in an acceptable amount of time. 
\par
This thesis also discusses the shortcomings of the new approach and suggest some ways to improve on those. The insights from the experiments show that there are still many opportunities for improvements and refinements with additional work in the future.
\end{abstract}

\begin{abstract*}
Deze thesis presenteert een nieuw traject-planningsalgoritme voor onbemande multirotor luchtvaartuigen (Unmanned Areal Vehicle, of UAV, in het Engels). Dit algoritme is uiterst schaalbaar en maakt gebruik van Gemengd  Geheeltallig Linear Programmeren (meestal vermeld met de Engelse term: Mixed-Integer Linear Programming, ofwel MILP). Eerdere systemen die MILP gebruikten voor trajectplanning hadden last van een slechte schaalbaarheid bij een lang traject of een groot aantal obstakels. De nieuwe aanpak in deze thesis kan trajecten plannen door 2D werelden die gebaseerd zijn op data van echte steden. Deze datasets zijn verschillende vierkante kilometers groot en hebben tot wel tienduizenden obstakels.
\par
Deze vooruitgang in performantie is mogelijk omdat het trajectplanningsprobleem wordt opgedeeld in een vele kleinere deelproblemen. Elk deelprobleem modelleert slechts een klein stuk van het gehele traject. Om het probleem in deelproblemen op te delen berekent het algoritme eerst een pad naar de doelpositie. Een pad, in tegenstelling tot een traject, is niet afhankelijk van tijd en houdt geen rekening met de dynamische eigenschappen van de UAV. De deelproblemen worden gegenereerd op basis van de bochten in dit pad, waardoor het aantal obstakels dat gemodelleerd moet worden in elk deelprobleem beperkt blijft.
\par
Het algoritme werd uitgetest op verschillende scenario's. Een aantal van die scenario's neemt plaats in een wereld die gebaseerd is op kaarten van San Francisco of Leuven. De experimenten tonen aan dat de schaalbaarheid van het algoritme vooral afhangt van de dichtheid van de spreiding van de obstakels. Ondanks dat kan het algoritme nog altijd steeds een traject plannen in een aanvaardbare hoeveelheid tijd, zelfs met de dichtheid van obstakels in de Leuven dataset die groter is dan nodig voor deze toepassing.
\par
De thesis bespreekt ook de beperkingen van deze nieuwe aanpak en stelt een aantal oplossingen voor. De inzichten die voortvloeien uit de experimenten tonen dat er nog altijd veel mogelijkheden zijn om het algoritme te verbeteren en verfijnen met wat extra werk in de toekomst.
\end{abstract*}

\listoffigures
\listoftables

\chapter{List of Abbreviations}
\section*{Abbreviations}
\begin{flushleft}
  \renewcommand{\arraystretch}{1.1}
  \begin{tabularx}{\textwidth}{@{}p{12mm}X@{}}
    2D   & Two dimensional \\
    GA   & Genetic Algorithm \\
    LP   & Linear Programming \\
    MAD   & Maximum Acceleration Distance \\
    MILP   & Mixed-Integer Linear Programming \\
    PRM   & Probabilistic Roadmap \\
    RRT   & Rapidly-Exploring Random Tree \\
    SF   & San Francisco \\
    UAV  & Peak Signal-to-Noise ratio \\
  \end{tabularx}
\end{flushleft}

\mainmatter
\begin{abstract}
Trajectory planning using Mixed Integer Linear Programming (MILP) is a powerful approach because vehicle dynamics and other constraints can be taken into account. However, it is currently severely limited by poor scalability.
This paper presents a new approach which improves the scalability regarding the amount of obstacles and the distance between the start and goal positions.
%
While previous approaches hit computational limits when the problem contains tens of obstacles, our approach can handle tens of thousands of polygonal obstacles successfully on a typical consumer computer. 
%
This performance is achieved by dividing the problem into many smaller MILP subproblems using two sets of heuristics. Each  subproblem models a small part of the trajectory. The subproblems are solved in sequence, gradually building the desired trajectory.
%
The first set of heuristics generate each subproblem in a way that minimizes its difficulty, while preserving stability. The second set of heuristics select a limited amount obstacles to be modeled in each subproblem, while preserving consistency.
%
To demonstrate that this approach can scale enough to be useful in real, complex environments, it has been tested on maps of two cities with trajectories spanning over several kilometers.
\end{abstract}

\section{Introduction}
Trajectory planning for multirotor UAVs is a complex problem because flying is inherently a dynamic process. Proper modeling of velocity and acceleration are required to generate a feasible trajectory that is both fast and safe, that is, the UAV should be able to effectively navigate corners while maintaining momentum. The fastest trajectory is not always the shortest one, since the UAV's velocity may be different. The UAV dynamics are often not the only constraints placed on the trajectory. Different laws in different countries also affect the properties of the trajectory. The operators of the UAV may also wish to either prevent certain scenarios or ensure that specific criteria are always met. \\
In this paper we present a scalable approach which is capable of generating fast and safe trajectories, while also being easily extensible by design. We model the trajectory planning problem as a Mixed Integer Linear Program (MILP). The trajectory is represented in discrete time steps where each step describes the UAV's dynamic state at that moment. 
% In MILP, the problem is modeled using mathematical equations as constraints. 
An objective function encodes one or more properties, like time or trajectory length, to be optimized. A general solver is then used to find the optimal solution for the problem. Because the problem is defined declaratively, additional constraints can easily be added.\\
We demonstrate our approach in 2D environments. We assume that all obstacles are polygons, static and known in advance. Our algorithm is designed for offline planning, ensuring that a feasible trajectory exists before the UAV starts executing its task. 
% These limitations were considered, but we decided to keep the problem simple and focus on demonstrating that the new approach does in fact work. 
% We believe that the approach should also be effective in 3D, although more work is necessary.\\
Other papers have used MILP for trajectory planning \cite{Schouwenaars2001}, their approaches could not be used to generate long trajectories through complex environments. Our main contribution is an approach which improves the scalability by dividing the problem into many MILP subproblems. Each subproblem models only a part of the trajectory. The subproblems are solved sequentially. A first set of heuristics uses a Theta* path to generate the subproblems. The second set of heuristics select which obstacles should be modeled in each subproblem, limiting the amount of obstacles that need to be modeled while ensuring that no collisions can occur.

%Our main contribution is a preprocessing pipeline which makes MILP trajectory planning more scalable. In our approach, we divide the trajectory into many smaller segments. Several steps of preprocessing collect information about the trajectory. This information is used to generate the smaller segments, as well as reduce the difficulty of each specific segment. \\
%The first step consists of finding an initial path using the Theta* algorithm. This trajectory does not take dynamic properties into account, making it faster to calculate. In the second step, the corners in this initial path are extracted and used to generate the segments. We define a corner to be a distinct change in the path's direction, with that change in direction being necessary because at least one obstacle is in-between the position where the turn starts and the goal position. The third step attempts to minimize the amount of obstacles that need to be modeled in each segment. A heuristic selects important obstacles for each segment. An obstacle is important if its absence could have a large impact on the trajectory. These obstacles are considered active and will be fully modeled in the MILP problem, ensuring that the vehicle will not collide with an active obstacle.\\ 
%To ensure that the trajectory does not intersect with the inactive obstacles, a safe area is constructed by a genetic algorithm. This safe area is constructed such that it does not contain inactive obstacles and is convex. The vehicle is constrained stay within the safe area at all times in the MILP problem. To avoid restricting the movement of the vehicle unnecessarily, the genetic algorithm attempts to maximize the size of the safe area.\\
Schouwenaars et al.\cite{Schouwenaars2001} were the first to demonstrate that MILP could be applied to trajectory planning problems. They used discrete time steps to model time with a vehicle moving through 2D space. Obstacles are modeled as grid-aligned rectangles. To limit the computational complexity, they presented a receding horizon technique so the problem can be solved in multiple steps. However, this technique is essentially blind and could easily get stuck behind obstacles. Bellingham\cite{Bellingham2002} recognized that issue and proposed a method to prevent the trajectory from getting stuck behind obstacles, even when using a receding horizon. However Bellingham's approach still scales poorly in environments with many obstacles. The basic formulations of constraints we present in this paper are extensions of the work by Bellingham.\\
Flores\cite{Flores2007} and Deits et al.\cite{Deits2015} do not use discretized time, but model continuous curves instead. This not possible using linear functions alone. They use Mixed Integer Programming with functions of a higher order to achieve this. The work by Deits et al. is especially relevant to this paper, since they also use convex safe regions to solve the scalability issues when faced with many obstacles.
%Several papers \cite{Fliess1995a, Hao2005, Cowling2007, Mellinger2011} show how the full state of a quadrocopter, including motor thrust, can be determined using only the 3D position of the vehicle's center of mass, the yaw and their derivatives. This demonstrates that, when the properties of a vehicle are accurately modeled, trajectory planners like the one in this paper need minimal post-processing to control a vehicle. Of course that does assume these planners can run in real time to deal with errors that inevitably will grow over time. Culligan \cite{Culligan2006} provides an online approach. However, their approach can only find a suitable trajectory for the next few seconds of flight and updates that trajectory constantly. \\
%More work has been done on modeling specific kinds of constraints or goal functions. For instance, Chaudhry et al. \cite{Chaudhry2004} formulated an approach to minimize radar visibility for drones in hostile airspace. However, none of these have really attempted to make navigating through a complex environment like a city feasible. The approach by Deits et al. \cite{Deits2015} could work, but did not explore the effects of longer trajectories on their algorithm.
\clearpage
\section{Modeling Path Planning as a MILP problem}
\label{section:modelingbasic}
\subsection{Introduction}
TODO: REWRITE
MILP is a form of mathematical programming, form of declarative programming. contrast with imperative programming: say what the solution looks like instead of how to get there. Mathematical programming: subset of declarative programming where problem is defined as mathematical problem. Solver computes result.\\
Big advantages: \\
- dont need to say how it is done, focus is on precisely modeling problem.\\
- flexible: can easily change rules: make problem more restrictive or relaxed, change the goal, etc ``just works''\\
- can be really fast thanks to work on solvers\\
\\
disadvatages:\\
- no free lunch (find reference): for anything more than very basic cases, solver cannot guarantee that a solution will be found any faster than random search\\
	--> cant rely on just the solver doing the work, problem needs to be stated in a way that guides solver in the right direction
	--> even when careful, as problem becomes more complex, solvers struggle\\
-  hard to understand: solver finds solution or not.
	-->Solution is optimal, but if it is not as expected it does not give any information why.
	-->when no solution: can show which constraint failed, but problem is not necessarily there. complex interplay between constraints is extremely hard to debug\\


\subsection{Terminology}
TODO: FILL OUT
variable
constraint
solution space
feasible region
solver (cplex)
convex

\subsection{Importance of integers and convexity}
TODO: REWRITE
linear programming one of the most simple forms of mathematical programming. Only linear functions allowed, all variables have a real domain. If feasible region exists: inherently convex. When a problem is convex, typically efficient solvers can be written (TODO: find source). This is the case for LP. LP problems can be solved reliably and quickly.
\\
Problem: only real variables is very limiting. Logical operators are impossible to model. Result is that not every problem can be modeled without integers. Adding integers seems like a minor change, but it no longer guarantees that the feasible region is convex. When the problem is no longer convex, it becomes intractible. An intractible problem is a problem for which we do not have a strategy that will on average find the solution faster than random guessing.
\\
Path planning when obstacles are present inherently needs integer variables and thus is an intractible problem. The goal of this thesis is to make this approach to path planning scalable. This will rely on two concepts://
- The solvers are on average not faster than random, but that's the average of all problems. We are only concerned with a very specific kind of problem. By being aware of how the solver works and simply experimenting with different representations, the problem can be solved much faster\\
- We can cheat. The problem can be greatly simplified and approximated. This means that the solution is no longer guaranteed to be optimal, but as long as it is good enough this is not a problem.\\


\subsection{Basic Path Planning MILP Model}

The path planning problem can be represented with discrete timesteps with a set of state variables for each epoch. The amount of timesteps determines the maximum amount of time the vehicle has in solution space to reach its goal. The actual movement of the vehicle is modeled by calculating the accelleration, velocity and position at each timestep based on the throttle (in each axis) and state variables from the previous time step.

\begin{figure}[h]
\begin{math}
time_0 = 0 \\
time_{t+1} = time_{t} + \Delta t,  \quad 0 \leq t < N \\ \\
\boldsymbol{pos}_0 = \boldsymbol{pos}_{start} \\
\boldsymbol{pos}_{t+1} = \boldsymbol{pos}_{t} + \Delta t * \boldsymbol{vel}_{t}  \quad 0 \leq t < N \\ \\
\boldsymbol{vel}_0 =\boldsymbol{vel}_{start} \\
\boldsymbol{vel}_{t+1} =\boldsymbol{vel}_{t} + \Delta t * \boldsymbol{acc}_{t}  \quad 0 \leq t < N \\ \\
\boldsymbol{acc}_{t} = \boldsymbol{throttle}_{t} * \boldsymbol{acc}_{max}  \quad 0 \leq t \leq N \\
\end{math}
\end{figure}

The problem also needs a goal function to optimize. In this model, the goal is to minimize the time before a goal position is reached. Optionally, there is also a goal velocity that needs to be matched when the vehicle reaches the goal. Reaching the goal constraints causes a state transition from not being finished to being finished. Modeling state transitions directly can be error-prone, so Lamport's\cite{Lamport1989} state transition axiom method was used. In this simple model it is still possible to model the state transition directly, but the goal of the thesis is to provide a flexible and extensible approach.


\begin{figure}[h]
\begin{math}
minimize \quad N - \mathlarger{\sum}_{t=0}^{t \leq N} fin_t \\
fin_0 = 1 \\ 
fin_{t+1} = fin_t \vee cfin_{t+1},  \quad 0 \leq t < N \\ \\
cfin_{pos,t} =  \mathlarger{\mathlarger{\bigwedge_{i = 0}^{i < Dim(\boldsymbol{pos}_t)}}} |pos_{t,i} - pos_{goal, i}| < pos_{tol},  \quad 0 \leq t \leq N \\ \\ \\
cfin_t =  cfin_{pos,t} \quad 0 \leq t \leq N
\end{math}
\end{figure}


The most challenging part of the problem is modeling obstacles. Any obstacle between the vehicle and its goal will inherently make the search space non-convex. Because of this, integer variables are needed to model obstacles. The most common way to do this is to use the ``Big M'' method to model a polygon. The size of the vehicle needs to be taken into account. Assuming the polygon is convex and the vertices of the polygon are listed in counter-clockwise order, the following constraints model an obstacle:

\begin{figure}[h]
OBSTACLE CONSTRAINTS \\
\end{figure}

\subsection{Characteristics of the basic model}
Nothing new in basic model, motivate changes based on flaws in the model. demonstration with simple scenario (benchmark and spiral) where basic model already struggles. Approximate amount of integer variables needed, use as representation of nonconvexity of problem. \\
Also show issue with time allowed: Need to allocate way more time than is needed, making the problem even harder to solve. Limiting time makes problem faster to solve, but cannot know without prior information.\\
Demonstrate issue with corners. Corners make execution time go up even as amount of integer variables stays the same. If no corners are needed, feasible space stays more convex even though integer values are same. Nuance that integer variables are only apprixmation of non-convexity, convexity is real property that matters.\\
Conclude: need to reduce amount of obstacles, need to reduce timesteps, limit amount of corners at a time (so problem becomes more convex). Dividing problem into smaller pieces tackles all these goals.

\clearpage
\section{Segmentation of the MILP problem}
\label{section:segment}
This section proposes a preprocessing algorithm to improve the scalability of MILP trajectory planning.\\ 
Assuming Hypothesis \ref{hyp:nonconvex} is true, a high degree of non-convexity around the trajectory causes slow solve times. There are two issues with this. The first issue is that the trajectory is the solution for the problem. If the solution is known, the problem does not need to be solved anymore. The second issue is that even if the optimal trajectory is known, the degree of non-convexity around that trajectory cannot be reduced by much since it already is the optimal trajectory. For instance, the Up/Down scenario in Figure \ref{fig:benchmarks-convex} cannot be made more convex without changing the original problem.\\
However, these issues can be resolved by relaxing the requirement of finding the optimal trajectory. The UAV operates in the real world, so a "good enough" trajectory will do. The MILP trajectory planning problem can be divided into many sub-problems, each of which solve a small part of the trajectory. By dividing the MILP problem into smaller sub-problems, the guarantee of optimality no longer holds. The optimal trajectory for each sub-problem will be found, but the combined trajectories may not be the optimal solution for the original problem. However, the degree of non-convexity can be limited in each sub-problem, which improves the scalability. \\
The non-convexity of a trajectory planning problem manifests itself as turns in the trajectory. Because of this, limiting the amount of turns in each sub-trajectory also limits the degree of non-convexity around the sub-trajectory. Limiting the amount of turns in each sub-problem is one of the goals for the preprocessing algorithm.\\
Before the algorithm can take turns into account when building the sub-problems, the turns in the trajectory need to be known. This trajectory is not known in advance, but this is not required. A faster path planning algorithm can be used to find a path from the start to the goal position. Unlike a trajectory, a path is not time-dependent and does not take the dynamics of the UAV into account. These simplifications make path planning algorithms typically much faster than trajectory planning algorithms. Despite the limitations of the path, the turns in the path do correspond to turns in the trajectory. However, it also means the trajectory may not be the fastest trajectory. Once again, optimality is sacrificed. \\

%\subsection{Introduction}
%The MILP model described in section \ref{section:modelingbasic} is sufficient to solve the trajectory planning problem for short flights with few obstacles. However, it scales poorly when the duration of the flight or the amount of obstacles is increased. Mixed-Integer programming belongs to the ``NP-Complete'' class of problems \cite{DBLP:conf/coco/Karp72}. This is a class of problems which is considered very hard to solve. As the amount of integer variables grows, the time needed to solve the problem increases exponentially. An integer variable is needed for every edge of every polygon, for every time step. By reducing both the amount of time steps needed and obstacles that need to be modeled, the execution time can be reduced dramatically. \\
%The key insight that allows my algorithm to scale well beyond what's usually possible is that the path trajectory not need to be solved all at once. If the trajectory planning problem can be split into many different subproblems, each subproblem becomes easier to solve. The solution for each subproblem is a small part of the final trajectory. By solving theses subproblems sequentially, the final trajectory can gradually be constructed. \\
%While diving the problem into subproblems does make things much easier to solve, it also has an important down side: Finding the fastest trajectory can no longer be guaranteed. Smaller subproblems make it easier to find a solution, but fundamentally the problem of finding the optimal trajectory is still just as hard. The necessary trade-off for better performance is that the optimal trajectory will likely not be found. Luckily, the optimal trajectory is often not required in navigation. A reasonably good trajectory will do.
%
%\subsubsection{The importance of convexity}
%While the worst case time needed to solve a MILP problem increases exponentially with the amount of integer variables, this is not the most useful way to measure the difficulty of a problem. Modern solvers are heavily optimized and are able to solve certain problems with many integer variables much faster than others. The key difference is the convexity of the solution space. Just like a circle is the solution space for ``all points a certain distance away from the center point'', the constraints used to model the trajectory planning problem form some geometric shape with a dimension for every variable. \\
%When only linear constraints with real values are used, the solution space will always be convex. It is this convexity that makes a standard linear program easy to solve. When integer variables are introduced, it is possible to construct solution spaces which are not convex. As the solution space becomes less and less convex, the problem becomes harder to solve. Integer variables can be seen as a tool which allows non-convex solution spaces to be modeled. When trying to improve the difficulty of a problem, the actual goal is making the problem more convex (or smaller, which always helps). Reducing the amount of integer variables is only a side effect. \\
%This insight is critical when determining how to divide the trajectory problem into smaller subproblems, and which obstacles are important for each subproblem.
\subsubsection{General Algorithm Outline}
\begin{algorithm}
\caption{General outline}
\label{alg:outline}
\begin{algorithmic}[1]
\State $T \leftarrow \{\}$ \Comment{The list of solved subtrajectories}
\State $path \leftarrow$ \Call{Theta*}{$scenario$}
\State $events \leftarrow$ \Call{FindTurnEvents}{$path$}
\State $segments \leftarrow$ \Call{GenSegments}{$path$, $events$}
\ForEach {$segment \in segments$}
\State \Call{UpdateStartState}{$segment$}
\State \Call{GenSafeRegion}{$scenario$, $segment$}
\State \Call{GenSubMILP}{$scenario$, $segment$}
\State $T \leftarrow T \cup \{$ \Call{SolveSubMILP}{} $\}$
\EndFor
\State $result \leftarrow $\Call{MergeTrajectories}{$T$}
\end{algorithmic}
\end{algorithm}

Algorithm \ref{alg:outline} shows the general outline of the algorithm. It consists of two phases. The first phase gathers information about the trajectory planning problem. A Theta* path planning algorithm is used to find an initial path (line 2). From this initial path, turn events are generated (line 3). These turn events mark where the trajectory will have to turn. \\
The seconds stage builds and solves "segments". Each segment represents a single sub-trajectory. The segments contain the information needed to build the corresponding MILP sub-problem, which can in turn be solved by the solver. First, each segment is generated (line 3) from the turns found in the previous step. Before the MILP sub-problem can be generated and solved (line 8-9), a heuristic selects several obstacles to be modeled in the problem. A genetic algorithm generates a safe region which is allowed with those selected obstacles only (line 7). To ensure a seamless transition between two consecutive sub-trajectories, the starting state for the UAV in the MILP current segment is updated to match the final state of the UAV in the previous segment (line 6). Finally, the sub-trajectories are merged together to form the final trajectory (line 11).

%The algorithm I developed consists of two phases.\\
%In the first phase, information is gathered about the trajectory planning problem. The Theta* algorithm is used to construct an initial path from the start to the goal position. Unlike a trajectory, a path is not time-dependent and does not take dynamic properties into account. This makes a path much easier to calculate than a trajectory. While the Theta* path is not suitable to describe the movements of a UAV, it contains useful information about the trajectory we want to calculate. More specifically, the turns in the Theta* path are used to generate path segments. Each path segment contains the information needed to construct a MILP subproblem. \\
%In the second phase, each path segment considered sequentially. First, for all but the first segment,


\subsection{Finding the initial path}
The first step in Algorithm \ref{alg:outline} is finding the Theta* path (line 2). This path will be used to divide the trajectory planning problem into segments. The MILP-problem generated from each of those segments needs an intermediate goal to get the UAV closer to the final goal position. These intermediate goals will be determined by the Theta* path. \\
This path is not only useful to guide the trajectory towards the goal. It is also lets the algorithm determine where the turns will be in the trajectory.

\subsubsection{A*}
Theta* is a variant of the A* path planning algorithm. In A*, the world is represented as a graph. Each possible position is represented as a node, with edges between nodes if one position can be reached from the other. The distance between connected positions is represented as a weight or cost on each edge. \\
Planning a path consists of picking a start and goal position. The A* algorithm will traverse the edges between nodes, keeping track of which edges it traversed to reach a certain node. When the algorithm reaches the goal node, the edges traversed to reach that node are the path from the start node to the goal node. \\
In this case, the world the UAV travels through is a continuous (2D) world. The graph for A* star is generated by overlaying a grid on the world. Nodes are placed at each intersection of the grid, as long as they are not inside obstacles. Each node is also connected to its neighbors by moving horizontally, vertically or diagonally along the grid. Figure \ref{fig:astar-grid} shows an example of this.
\begin{figure}
\centering
\includegraphics[width=0.5\textwidth]{astar-grid}
\caption{An example of how a grid is used to build the graph for the path finding algorithm. Each point is a node on the graph. If two points are connected by a line, their nodes in the graph also are connected by an edge. Diagonal edges are not shown here for clarity.}
\label{figure:astar-grid}
\end{figure}

\subsubsection{Reason for Theta*}
A* finds the shortest path through the graph. However, this graph is only an approximate representation of the actual continuous world. An A* path will only travel along the edges of the graph. This means that the path can only travel horizontally, vertically and possible diagonally. If the shortest path between two points is at another angle, the A* path will contain zig-zags or detours because it is limited to traveling along the grid. \\
Theta* solves this problem. It is nearly identical to A*, but it allows the path to travel at arbitrary angles. It still traverses the graph using the edges between nodes, but does not restrict the path to only following those edges.

\begin{figure}
\includegraphics[width=0.5\textwidth]{a_theta_star_comp3}
\caption{The red line shows a typical A* path, compared to the path found by Theta*. The gray rectangle is an obstacle.}
\label{figure:thetastarcompare}
\end{figure}


\subsubsection{Theta* implementation}
\begin{algorithm}
\caption{Theta* Implementation}
\label{alg:thetastar}
\begin{algorithmic}[1]
\State $g(v_{start}) \leftarrow 0$
\State $parent(v_{start}) \leftarrow null$
\State $queue \leftarrow \emptyset$
\State $queue$.Insert($v_{start}$, $g(v_{start}) + c(v_{start}, v_{goal})$)
\While{$queue \neq \emptyset$}
	\State $s \leftarrow queue$.Pop()
	\If{$s = s_{goal}$}
		\State return "path found"
	\EndIf
	\ForEach{$s' \in ~ $GenerateNeighbors($s$)}
		
	\EndFor
\EndWhile
\end{algorithmic}
\end{algorithm}
TODO: paraphrased! CITE!!
Algorithm \ref{alg:thetastar} shows how Theta* is implemented. It uses the following elements:
\begin{itemize}
\item the g-value $g(s)$ is the length of the shortest path between the start node and $s$.
\item a function $c(s,s')$ which returns the distance between node $s$ and $s'$.
\item a function $parent(s)$ which returns the node before $s$ in the path. When the parent of a node is $null$, it is either not part of the path or the first node of the path.
\item a priority queue $queue$. This is a queue of nodes to expand next. Each node $s$ is added with a value $x$ using the $queue.$Insert($s$,$x$) method. The $queue$.Pop() method removes and returns the node $s$ with the lowest value $x$.
\item a set $expanded$ which contains all nodes which have already been expanded.
\item a function GenerateNeighbors($s$) which generates and returns the neighbors of node $s$. These are the neighboring positions
\end{itemize}
When the algorithm initializes, the g-value of the start


\subsubsection{Theta* optimizations}
heuristic \\
indexed obstacles \\
possible position: first fuzzy collision, after: LOS check with last point \\
line of sight: first bounding box overlap, only after LOS \\

\subsubsection{Scalability of Theta*}
TODO: REWRITE... KEEP?
Theta* provides the information that makes it possible to make MILP fast, but Theta* in itself is intractable. Why is this not simply moving the issue?\\
Theta* is indeed intractable, but still much better than MILP. Holomomic vs non-holonomic. Concept of navigation mesh as part of map, often used in video games to allow AI to navigate large worlds. Also same concept behind PRM. Only needs to pre precomputed once per map. Can serve as heuristic for Theta*, or completely replace it.


\subsection{Detecting corner events}
With an initial path generated, the next problem is dividing it into segments. Dividing the path into equal parts presents problems, because when solving each segment, the solver has no knowledge of what will happen in the next segment. This is especially problematic when the vehicle needs to make a tight corner. If the last segment ends right before the corner, it may not be possible to avoid a collision. Because of this, corners need to be taken into account when generating the segments. The right image in figure \ref{fig:pre-1-2} shows the transitions between segments as green circles.\\
In Euclidian geometry, the shortest path between two points is always a straight line. When polygonal obstacles are introduced between those points, the shortest path will be composed of straight lines with turns at one or more vertices of the obstacles. The obstacle that causes the turn will always be on the inside of the corner. This shows why corners are important for another reason: They make the search space non-convex. For obstacles on the outside of the corner it is possible to constrain the search space so it is still convex, but this is not possible for obstacles on the inside of a corner.\\
Because of these reasons, isolating the corners from the rest of the path is advantageous. With enough buffer before the corner, the vehicle is much more likely to be able to navigate the corner successfully. It also means that the computationally expensive parts of the path are as small as possible while still containing enough information for fast navigation through the corners.
\\
The reason for using Theta* becomes clear now. Every single node in the path generated by the algorithm is guaranteed to be either the start, goal or near a corner. A corner can have more than one node, so nodes which turn in the same direction and are close to each other are grouped together and considered part of the same corner. For each corner, a corner event is generated.

\subsubsection{Algorithm Implementation}
code \\
parameter tolerance, link to acc distance\\
\subsubsection{Tight Coupling with Theta*}
This algorithm relies on assumptions from Theta*, so if algorithm changes, this will need to be updated.


\begin{figure}[!t]
    \centering
    \begin{subfigure}[t]{0.47\textwidth}
        \includegraphics[width=\textwidth]{img/pre1}
    \end{subfigure}
    \hfil
    \begin{subfigure}[t]{0.47\textwidth}
        \includegraphics[width=\textwidth]{img/pre2}
    \end{subfigure}
    \caption{The left image shows the results after the Theta* algorithm has executed. The blue shapes are obstacles, while the gray line is the Theta* path. The right image shows the results after the path is segmented. Extra nodes have been added to the Theta* path, as marked by the green circles. These circles depict the transitions between segments.}\label{fig:pre-1-2}
\end{figure}

\begin{figure}[!t]
    \centering
    \begin{subfigure}[t]{0.47\textwidth}
        \includegraphics[width=\textwidth]{img/pre3}
    \end{subfigure}
    \hfil
    \begin{subfigure}[t]{0.47\textwidth}
        \includegraphics[width=\textwidth]{img/pre4}
    \end{subfigure}
    \caption{The left image shows the result after the genetic algorithm has executed. The obstacles in red have been selected to be modeled in the MILP problem. The dark grey shape is the convex allowed region generated by the genetic algortihm. Note how it does not overlap with any of the blue obstacles. The right image shows the final result. The trail of circles show the path of the vehicle up to the current time step, which is represented by the filled circles. The red and yellow colors depict the same information as in figure \ref{fig:obs}}\label{fig:pre-3-4}
\end{figure}

\subsection{Generating path segments}
\begin{algorithm}
\caption{Generating the segments}
\label{alg:segments}
\begin{algorithmic}[1]
\Function{GenSegments}{$path$, $events$}
\State $segments \leftarrow \{\}$
\State $catchUp \leftarrow true$
\State $lastEnd \leftarrow path(0)$
\For {$i \gets 0, |events| - 1 $}
\State $event \leftarrow events(i)$
\If{$catchUp$}
	%\State $segStart \leftarrow $ \Call{ExpandBackw}{$event.start$}
	%\State \Call{AddSegments}{$lastSegEnd$, $segStart$}
	%\State $lastSegEnd \leftarrow segStart$ 
	\State expand $event.start$ backwards
	\State add segments from $lastEnd$ to $event.start$
	\State $lastEnd \leftarrow event.start$
\EndIf
\State $nextEvent \leftarrow events(i+1)$
\If{$nextEvent.start$ is close to $event.end$}
	\State $mid \leftarrow$ middle between $event$ \& $nextEvent$
	\State add segment from $lastEnd$ to $mid$
	\State $lastEnd \leftarrow mid$
	\State $catchUp \leftarrow false$
\Else
	\State expand $event.end$ forwards
	\State add segment from $lastEnd$ to $event.end$
	\State $lastEnd \leftarrow event.end$
	\State $catchUp \leftarrow true$
\EndIf
\EndFor
\State add segments from $lastEnd$ to $path(|path|-1)$
\Return $segments$
\EndFunction
\end{algorithmic}
\end{algorithm}



%These corner events are in turn grown outwards to cover the approach and departure from the corner. How much depends on the maximum acceleration of the vehicle. As a rule of thumb: If the vehicle can come to a complete stop from its maximum speed before the corner, it can also successfully navigate that corner. When corners appear in quick succession, their expanded regions may overlap. In that case, the middle between those corners is chosen. Long, straight sections are also divided into smaller path segments.\\
%One of the main goals of segmenting the path is to reduce the amount of obstacles. This means that every segment has a set of obstacles associated with it, being the obstacles that need to be modeled in the optimization step. Not only the obstacle that ``causes'' the corner is important, but obstacles which are nearby are important as well. Obstacles on the outside of the corner also may play a role in how the vehicle approaches the corner. To find all potentially relevant obstacles, the convex hull of the (Theta*) path segment is calculated and scaled up slightly. Every obstacle which overlaps with this shape is considered an active obstacle for that path segment. The convex hull step ensures that all obstacles on the inside of the corner are included, while scaling it up will cover any restricting obstacle on the outside of the corner.

\subsubsection{Segment Generation Algorithm}
code \\
approachmargin \\
maxtime \\
step by step figures with explanation? 
\subsubsection{Segment data}
relevant points: pre path + stop point going in + stop point at finish. image with points clearly labeled \\
convex hull: quickhull. source + image for demo \\
note: only when about to be solved!
\subsection{Generating the active region for each segment}
One of the main goals of segmenting the path is to reduce the amount of obstacles. Every segment has a set of active obstacles associated with it, being the obstacles that need to be modeled for the solver. Not only the obstacle that ``causes'' the corner is important, but obstacles which are nearby are important as well. Obstacles on the outside of the corner also may play a role in how the vehicle approaches the corner. To find all potentially relevant obstacles, the convex hull of the (Theta*) path segment is calculated and scaled up slightly. Every obstacle which overlaps with this shape is considered an active obstacle for that path segment. The convex hull step ensures that all obstacles on the inside of the corner are included, while scaling it up will cover any restricting obstacle on the outside of the corner.\\
The inactive obstacles also need to be represented. To do this, a convex polygon is constructed around the path. This polygon may intersect with the active obstacles (since they will be represented separately), but may not intersect any other obstacle. The polygon is grown using a genetic algorithm. Genetic algorithms are inspired by natural selection in biology. A typical genetic algorithm consists of a population of individuals, a selection strategy and one or more operators to generate offspring. In each generation, the operators are applied on the population to produce offspring. These operators usually have a random element and are responsible for exploration of the search space. The selection strategy determines, often based on a fitness function, which individuals survive and form the population for the next generation. Selection is responsible for convergence towards fitter individuals, limiting how much of the search space is evaluated. \\

\subsubsection{Implementation of the genetic algorithm}
\begin{algorithm}
\caption{Genetic Algorithm}
\label{alg:ga}
\begin{algorithmic}[1]
\Function{GenActiveRegion}{$scenario$, $segment$}
\State $pop \leftarrow $ \Call{SeedPopulation}{}
\For {$i \gets 0, N_{gens} $}
\State $pop \leftarrow pop \cup $ \Call{Mutate}{$pop$}
\State \Call{Evaluate}{$pop$}
\State $pop \leftarrow $ \Call{Select}{$pop$}
\EndFor
\Return \Call{BestIndividual}{$pop$}
\EndFunction
\Function{Mutate}{$pop$}
\ForEach {$individual \in pop$}
\State add vertex with probability $p_{add}$
\State OR remove vertex with probability $p_{remove}$
\ForEach {$gene \in individual.chromosome$}
\State randomly move vertex by at most $\Delta_{nudge}$
\If{new polygon is legal}
\State update polygon
\Else
\State try again at most $N_{attempts}$ times
\EndIf
\EndFor
\EndFor
\Return \Call{BestIndividual}{$pop$}
\EndFunction
\end{algorithmic}
\end{algorithm}

Alg. \ref{alg:ga} shows our implementation. In our implementation, each individual in the population represents a single legal polygon. A legal polygon is convex, does not self-intersect, does not overlap with inactive obstacles and contains every node in the Theta* path for that specific segment. The latter requirement prevents the polygon from drifting off. Each individual has a single chromosome, and each chromosome has a varying number of genes. Each gene represents a vertex of the polygon.\\
The only operator is a mutator (line 4). Contrary to how mutators usually work, the mutation does not change the original individual. This means that the every individual can be mutated in every generation, since there is no risk of losing information. This mutator can add or remove vertices of the polygon by adding or removing genes (lines 12-13), but only if the amount of genes stays between $N_{genes,min}$ and $N_{genes,max}$. The mutator attempts to nudge every vertex/gene to a random position at most $\Delta_{nudge}$ away (line 15). If the resulting polygon is not legal, it retries at most $N_{attempts}$ times (line 16-19). \\
Tournament selection is used as the selector, with the fitness function being the surface area of the polygon (line 5-6). Fig. \ref{fig:pre-comp} Shows the active obstacles in yellow and red, as well as the polygon generated by the genetic algorithm in dark gray.\\


library \\
tournament selection \\
offspring generation: mutate only \\
initial population \\
chromosome/gene description \\
fitness function \\
validation \\
parameters \\
obstacle selection based on path length? \\



\clearpage
\section{Extensions}
\label{section:extensions}

\subsection{Indicator constraints}
Big M method: functional, but solvers struggle with large value of M. Low value of M -> may cause wrong behavior. Especially obvious when near vertical line. \\
Can approximate near vertical line as vertical line, but other solution is integer constraint from cplex. serves same role as big M method, but can't fail, is clearer and provides more information to solver.

\subsection{Corner cutting}
ref \cite{Richards2015}
implemented simple version
TODO: implement more complex version
compare no vs simple vs complex

\subsection{Max time}
limit execution time, allows for clean failure if it takes to long and backtrack.
\subsection{Max delta}
Allows for earlier return. Limited usefulness?

\subsection{Abs speed}


\subsection{Max speed}

\subsection{Min speed}

\subsection{Backtracking}
implement better!

\clearpage
\section{Analysis and Results}
\label{section:analysis}

\subsection{Scenarios}
- Benchmark: a lot of corners with minimal amount of obstacles\\
- Spiral: Also many corners, minimal distance\\
- SF: real world grid. Many obstacles, corners nearly always 90 degrees. Very predictable for algorithm\\
- Leuven: real world and irregular. Even more obstacles. Very unpredictable\\

To test the complete algorithm, several different scenarios have been used. Each scenario has unique characteristics and was tested with two different problem sizes. Theta* is always executed with a grid size of 2m and each time step has a duration of 200ms. All tests were executed on an Intel Core i5-4690k running at 4.4GHz with 16GB of 1600MHz DDR3 memory. The reported times are averages of 5 runs. The machine runs on Windows 10 using version 12.6 of IBM CPLEX. Figure \ref{fig:scenarios} shows these scenarios visually. Table \ref{table:results} shows detailed information about the scenarios and execution times.

\subsubsection{Up/Down Scenario}
The first test scenario has very few obstacles, but lays them out in a way such that the vehicle needs to slalom around them. The small scenario has only 5 obstacles, while the larger one has 9. This is a very challenging scenario for MILP because every obstacle has a large impact on the path. Without segmentation on the version of the scenario, the solver does not find the optimal path within 30 minutes. If execution is limited to 10 minutes, the best solution it finds takes 26.0s to execute by the vehicle. That is less than a second faster than the segmented result while it took more than 20 times more execution time to find that solution. For the larger scenario with 9 obstacles, the solver could not find a solution within 30 minutes. This scenario clearly shows the advantages of segmentation, even if there only are a few obstacles.

\subsubsection{San Francisco Scenario}
The San Francisco scenario covers a 1km by 1km section of the city for the small scenario, and 3km by 3km section for the large scenario. All the obstacles in this scenario are grid-aligned rectangles laid out in typical city blocks. Because of this, density of obstacles is predictable. This scenario showcases that the algorithm can scale to realistic scenarios with much more obstacles than is typically possible with a MIP approach. With these constraints, parameters and hardware, the path can be planned faster than the vehicle can execute it.

\subsubsection{Leuven Scenario}
The Leuven scenario also covers both a 1km by 1km and 3km by 3km section, this time of the Belgian city of Leuven. This is an old city with a very irregular layout. The dataset, provided by the local government\footnote{\url{https://overheid.vlaanderen.be/producten-diensten/basiskaart-vlaanderen-grb}}, also contains full polygons instead of the grid-aligned rectangles of the San Francisco dataset. While most buildings in the city are low enough so a UAV could fly over, it presents a very difficult test case for the path planning algorithm. The density of obstacles varies greatly and is much higher than in the San Francisco dataset across the board. The algorithm does slow down, but it is still fast enough for offline planning. As visible in figure \ref{fig:pre-3-4}, there are many obstacles clustered close to each other, with many edges being completely redundant. For a real application, a small amount of preprocessing of the map data should be able to significantly reduce both the amount of obstacles as the amount of edges. 

\begin{figure}
	\centering
	
	\begin{subfigure}[t]{0.74\textwidth}
        		\includegraphics[width=\textwidth]{img/benchmarkfull}
        		\caption{}
	\end{subfigure}
		
	\begin{subfigure}[t]{0.70\textwidth}
        		\includegraphics[width=\textwidth]{img/SF}
        		\caption{}
	\end{subfigure}	
	
	\begin{subfigure}[t]{0.70\textwidth}
        		\includegraphics[width=\textwidth]{img/leuven}
        		\caption{}
	\end{subfigure}
        
    \caption{These are the three different worlds which were tested. The top image shows the large Up/Down scenario. The middle image shows the small San Francisco scenario. Note how the obstacles are only grid-aligned rectangles laid out in a grid pattern. The bottom image shows the small Leuven scenario. The obstacles are polygons and distributed in a much more irregular pattern compared to the San Francisco scenario.}\label{fig:scenarios}
\end{figure}


\begin{figure}
\begin{tabular}{ l l l l l l l l l }
 scenario & \# obstacles & world size & path length & \# segments & Theta* time & Gen. Al. time & MILP time & score \\ 
 \hline
Up/Down Small & 5 &  25m x 20m & 88m  & 7 & 0.09s & 1.10s & 20.8s & 26.6s\\
Up/Down Large & 9 & 40m x 20m &  146m & 11 & 0.14s & 1.62s & 40.1s & 43.6s \\
SF Small & 684 & 1km x 1km & 1392m & 28 & 2.04s & 9.56s & 59.2s & 105.7s \\
SF Large & 6580 & 3km x 3km & 4325m  & 84 & 18.14s & 18.21s & 231s & 316.0s\\
Leuven Small & 3079 & 1km x 1km & 1312m & 29 &  2.29s & 29.83s & 152s  & 95.9s \\
Leuven Large & 18876 & 3km x 3km & 3041m & 61 & 18.14s & 83.69s & 687s & 217.6s \\

\end{tabular}
\caption{The experimental results for the different scenarios}
\label{table:results}
\end{figure}
\subsection{Convexity of Search Space}
The convexity of the search space plays a large role in the difficulty of a MILP problem. My approach is built entirely around the idea of keeping the search space as convex as possible in each segment. To analyze  the importance of convexity, I tested three scenarios without any form of preprocessing. These scenarios are designed so the optimal trajectory has roughly the same duration in each scenario. The scenarios also each have 5 grid aligned rectangles as obstacles.\\
As a result, MILP problems for all three scenarios have nearly identical amounts of integer variables. The difference between these scenarios is the convexity of the search space. \\
In the first scenario, the obstacles are not in the way of the UAV. The UAV can move in a straight line from the start position to its goal. In the second scenario, two of the obstacles are slightly in the way. The UAV is forced to make slight turns near the start and end of the trajectory. However, most of the trajectory is still straight. In the third scenario, the vehicle has to slalom around every obstacle.\\

\subsubsection{Results}
TODO: ADD RESULTS\\
TODO: ADD IMAGES \\
\subsubsection{Interpretation}
Even though all scenarios have nearly identical amounts of time steps, integer variables and obstacles, the solve time varies greatly. This confirms the core assumption behind my approach: The convexity of the search space plays a larger role in the performance than the amount of integer variables.





\subsection{Genetic Algorithm Parameters}
tweak parameters...




\subsection{Agility of the UAV}
The properties of the segments strongly rely on the agility of the UAV. The size of the segments is determined by the maximum acceleration distance of the UAV. This is the distance the UAV needs to accelerate from zero to its maximum velocity, or decelerate from the maximum velocity to zero. \\
In this experiment, I tested relation between the maximum velocity and the maximum acceleration of the UAV. I used the large Up/Down scenario, the small San Francisco scenario and the small Leuven scenario. For each scenario I tested nine configurations of the vehicle: Every combination between three different maximum velocities and three different maximum accelerations.\\
TODO: add \# segments

\begin{figure}
\includegraphics[width=\textwidth]{img/agility1}
caption{}
\label{fig:agility-times}
\end{figure}
Even though some combinations did not finish for the larger scenarios, the results are relatively consistent across all three scenarios:\\
Within each speed category, a higher acceleration always reduces solve times. This is as expected, because a higher acceleration will reduce the expansion distance for the segments, making them both shorter in time as well as smaller so less obstacles are expected to be modeled.\\
Increasing the velocity for the low and medium acceleration tends to make things slower.
Low velocity and high acceleration always score well







\subsection{Cornercutting}
A trajectory that allows corner cutting cannot be considered safe. However, additional constraints are required to prevent this from happening. This experiment attempts to measure the impact of the corner cutting prevention.


\subsubsection{Result}
\begin{figure}
CORNER CUT
% \includegraphics[width=\textwidth]{img/agility1}
caption{}
\label{fig:corner-data}
\end{figure}

The performance impact caused by preventing corner cutting seems minimal (TODO: more precise!).



\subsection{Linear approximation}
The velocity and acceleration of the UAV are limited to some finite value. Because both of those quantities are vectors, that maximum can only be approximated with linear constraints. More constraints are needed to model this more accurately which can allow for faster solutions. However, more constraints also have a performance cost. This experiment analyses the trade-off that needs to be made.
\begin{figure}
LINEAR APPROX
% \includegraphics[width=\textwidth]{img/agility1}
caption{}
\label{fig:linear-approx-data}
\end{figure}


\subsection{Time step size}
The time step size determines how many time steps are used in each MILP problem. A smaller step sizes allows for better results, but at a performance cost

\begin{figure}
TIME STEP
% \includegraphics[width=\textwidth]{img/agility1}
caption{}
\label{fig:timestep-data}
\end{figure}
The time step size has a dramatic impact on performance. For the benchmark scenario, each increase in the amount of time steps resulted in around 14x the computation time. For the San Francisco scenario, that is around 6-7x. In both cases the trajectory time improved with more time steps as well, although the difference is smaller when going from 5fps to 10fps compared to 2fps to 5fps.


\subsection{Stability}

\subsection{Max Time}

\subsection{Issues}

\subsubsection{Active region}
often good results, but sometimes cuts corner too close

\subsubsection{Segment transitions}
small epsilon -> uav needs to slow down, forces trajectory very close to path
large epsilon + line -> uav tends to curve and still slow down close to transition, cause unknown?
						-> cause: needs to be able to stop?                        
occasional failures: jerk only? TEST!!
transition itself is weak point of approach. Possible way to solve: resolve transition area based on a bit before and after transition

\subsubsection{Maximum time for segment}
Finding maximum time for a segment needs to overestimate: more time steps cause longer solve times
Solution: generate path at lower temporal resolution to find accurate time needed. Could improve solve time





% \subsection{Safety}
% guaranteed by constraints, however:\\
% no incentive to stay away from walls. Can increase vehicle size, but prevents occasionally getting closer for a short while.\\
% \subsection{Stability and Performance}
% -maxtime (max length of segment)\\
% - approach margin: larger -> better approach, longer solve time\\
% - tolerance: larger -> when corners are close, merge faster: more execution time\\
% - position tolerance: larger -> smoother path, strays further from preprocessed path so needs to backtrack more often\\
% - backtracking effect?\\
% -can quickly recalculate based on measured state? IMPLEMENT WARM START EXPERIMENT?\\


\section{Discussion}
The results show a large improvement in scalability in certain realistic scenarios, but the choice of those scenarios has drastically impacted the algorithm I have developed. \\
During development, I have always used realistic approximations of the capabilities of multirotor UAVs. Those can reach high accelerations but have relatively low maximum velocities compared to what fixed-wing aircraft can achieve. This makes those vehicles very agile, which is one of the contributing factors to their recent popularity. \\
Assuming this agility means that my algorithm cannot be applied to UAVs which do not have that property. One of the properties my algorithm uses often is the maximum acceleration distance, which is the distance in which the UAV can always come to a complete stop. This works fine with multirotor UAVs, but is a meaningless concept when dealing with fixed-wing UAVs which cannot stop at all during flight. \\
However, those kinds of low-agility UAVs are unlikely to be deployed at low altitudes in dense city centers exactly because they lack agility. Even with perfect planning, cities are very unpredictable places. An multirotor UAV is much more likely to be able to safely react to an unexpected obstacle than a fixed-wing UAV. \\
While I picked out fixed-wing UAVs as an example, the same arguments hold for any kind of UAV that either cannot hover or has low agility. Some UAVs may be able to hover, but are not very agile due to a high maximum velocity and low acceleration. In this case, the agility can be improved by reducing the maximum velocity.\\ \\

Another point that warrants attention are the transition between segments. In some cases, the UAV may end a segment in a state which causes issues in the next segment. This may cause the next segment to not be solvable, or result in a strange and undesirable trajectory. Overlapping the segments does help, but the algorithm can still not guarantee that the next segment will be feasible. Backtracking does guarantee that the next segment will be feasible, but at a great computational cost.\\
This is partly due to active region generated by the genetic algorithm. Often the results are good, but in some cases the active region is very restrictive. The algorithm itself could certainly be improved, but it will be hard to guarantee a good result. Forcing the region to be significantly large solves the issue, but that comes at computational cost that may be too large.\\

However, I do believe that these transition issues can be solved. One of the next extensions I would try is solving the MILP problem first with a higher time step size. Using this to roughly solve the current and next segment to provide a suitable goal state (including velocity) for the first segment would ensure a good start for the next segment as well. This may also allow for a lower approach margin multiplier, because the proper approach is already determined. \\

Another extension I would try is using moving to Mixed Integer Quadratic programming. Linear approximations to limit the length of vectors works, but it also introduces artifacts into the path. Increasing the amount of constraints that model the norm helps minimize the impact, but comes at a performance cost. Stating those constraints directly as a quadratic function would reduce the amount of constraints needed per time step. This could improve performance, and entirely eliminates the artifacts. Especially when the problem is also extended to 3D, since the this would require the linear approximation of a sphere instead of a circle. \\

The obvious next step is extending this approach to 3D. The extra degree of freedom will likely come at a significant performance penalty, so this was not attempted during the thesis. One of the likely difficulties with the preprocessing as presented is that it treats all dimensions the same. This is fine for the horizontal dimensions, but due to gravity, movements the vertical dimension have different characteristics. The maximum acceleration of the UAV can no longer be assumed to be the same in all directions.\\
A possible mitigation to the increasing complexity of obstacles may be using a "2.5D" representation. A 2.5D obstacle is a 2D obstacle which also has a height value. This would only need one additional integer variable per obstacle to model. In a city scenario, this may be an acceptable approximation. \\
% \subsection{Other weaknesses}
% waviness\\
% bad approaches to corners\\
% transitions between segments\\\\

% eigen abs:\\
% slack -> abs = val\\
% -slack -> abs = -val\\\\

% --> 2 vars voor abs\\\\

% abs norm:\\
% speed >= dot product[i]\\
% slack[i] -> speed = dot product[i]\\\\

% --> speed: num points / 4 vars\\\\

% norm:\\
% slack[i] and absslackx -> circle x\\
% slack[i] and - absslackx -> - circle x\\\\

% --> norm: geen extra vars   \\
\clearpage
\chapter{Discussion}
\label{section:discussion}
The main focus points during this these were the performance and stability of the algorithm. Most decision were made with either performance or stability in mind, and often both. The performance and stability of the algorithm are discussed in section \ref{subsec:disc-perf} and \ref{subsec:disc-stab} respectively. Some important parameters and their impact on the algorithm are also discussen in \ref{section:imp-params}.
\section{Performance}
\label{subsec:disc-perf}
On the performance side, it is clear that the new algorithm with preprocessing is much faster than solve the pure MILP problem without preprocessing. Comparing the new algorithm to the pure approach is very difficult since the challenging scenarios for the new algorithm simply cannot be solved with the pure approach.
\par
When it comes to scalability, there are few noteworthy observations to make. 
\subsection{Path Length Scalability}
The first is that the time needed to solve each MILP subproblem does not depend on the length of the trajectory or the size of the world. Accounting for variations due to obstacle density, the average MILP solve time for scenarios using the same dataset (San Francisco or Leuven) are very similar. Since the amount of segments scales linearly with the path length, the MILP part of the algorithm also scales linearly with the length of the initial path. This is in stark contrast with the exponential worst-case performance of the pure MILP approach\\
However, this exponential complexity has not been eliminated. It has been shifted to the initial path planning algorithm, Theta*. This algorithm still has exponential worst-case complexity with respect to the length of the path. While Theta* does limit the scalability regarding the size of the world, it is a much easier problem to solve. It is part of the A* family of path planning algorithms which have been the subject of a large body of research.
\par
The algorithm separates the "routing" aspect from the trajectory planning aspect of the problem. The exact properties of the initial path are not very important. What matters is that it determines where and when to turn. By the time the MILP solver runs, the navigation aspect of the problem has already been solved. The MILP solver only needs to find a viable trajectory. The two aspects of the problem are solved separately, making both of them easier to tackle.
\par
I believe that identifying that these two aspects can be solved separately is the key insight that made the performance improvements possible. Solving these aspects separately means that the optimal trajectory is unlikely to be found. However, at this point that seems like a necessary sacrifice for long term trajectory planning through complex environments. I am not aware of any algorithm that scales as well as my algorithm does, and also finds the optimal trajectory.

\subsection{Obstacle Density Scalability}
The second observation is that the density of the obstacles plays a large role in the scalability of the algorithm. The San Francisco and Leuven scenarios are very similar, except for their obstacle density. The Leuven scenario has a significantly higher density of obstacles and is also much harder to solve. Without preprocessing, the scalability of the MILP problem is limited by the total amount of obstacles. Because each MILP subproblem in my algorithm is roughly the same size, the total amount of obstacles is no longer the limiting factor. The density of the obstacles is the limiting factor in my algorithm.
\par
The Leuven scenarios can still be solved in an acceptable amount of time, but I do not believe that this will be the case if the density is increased even more. Luckily, the Leuven dataset is more detailed than it needs to be. Each building in the city is represented separately, even when multiple buildings are connected. It should be possible to reduce the obstacle density substantially with a minimal amount of effort. Figure \ref{fig:leuven-dense2} shows a dense region in the Leuven dataset where there is a lot of room for improvement.

\begin{figure}[h]
	\centering
	\includegraphics[width=0.5\textwidth]{leuven-dense}
	\caption{One of the denser regions in the Leuven dataset}
	\label{fig:leuven-dense2}
\end{figure}

Given that the Leuven dataset is so unoptimized for this purpose and the algorithm still completes on average on the order of seconds per segment, I believe that the scalability with regards to the obstacle density is acceptable.

\subsection{UAV Agility} 
The last observation is the importance of the agility of the UAV. The algorithm was developed with high-end consumer to professional grade multirotor UAVs in mind. These are very agile vehicles capable of impressive feats of acrobatics when properly piloted. This agility is one of the assumptions this algorithm is based on. The UAV must be able to hover and accelerate quickly.
\par
The results from the UAV agility experiment in section \ref{subsec:agility} show that these assumptions are indeed a critical part of the performance of the algorithm. The algorithm fails when faced with UAVs with an (unreasonably) low agility. 
\par
This reliance on agility is one of the factors that made the dramatic improvement in performance possible, but it also limits the applicability of the algorithm. However, the goal of this thesis was not to developed a general algorithm. The algorithm performs well for reasonable estimates of the agility of a multirotor UAV. On top of that, the agility of any UAV can be increased by limiting its maximum velocity\footnote{Limiting the maximum velocity of UAVs with a low acceleration when navigating through a city seems like a wise decision anyway. Such a UAV would not be able to react to unexpected obstacles quickly, so having it fly at a high velocity through the city seems like a dangerous proposition.}.


\section{Stability}
\label{subsec:disc-stab}
For the algorithm to be useful, it must be stable. The first aspect of stability is whether or not it can find a solution. If the algorithm is capable of finding a solution, it should find that solution every time. It should also be able to solve problems with a similar difficulty as well. The second aspect is that the solution for the same problem should always be similar. There should be no large differences in the trajectory scores when the same problem is solved multiple times, nor should there be a large difference between very similar problems. This is also applies to the execution time. The execution time for similar problems should also be similar without large variations.
\par
My algorithm can find a solution most of the time. Due to what I believe to be a bug, it occasionally fails to find a solution. I was not able to fully understand why the bug occurs, but I believe it can probably be fixed.  
\par
The stability of the trajectory scores are excellent. All trajectories found are scored within a few percentage points of each other. When it comes to the execution time, there is more variation. However, with a standard deviation is around 10-15 \% of the mean execution time, I believe that the stability is still acceptable for offline trajectory planning.
%The general performance results show that the algorithm is capable of planning a long trajectory through complex environments. Even for the smallest scenario, the solver struggles to find a trajectory for a pure MILP problem without preprocessing. With my algorithm, the UAV can successfully navigate an entire city. \\
%
%The stability experiment shows that, for the most part, my segmentation approach preserves stability. There are occasional failures which seem to be caused by a bug. However, assuming this bug can be fixed the stability of the algorithm 
%
% Overall I am not entirely satisfied with the stability of my current implementation, however I do believe the small issues around segment transitions can be resolved. \\\\
%
%The results show a large improvement in scalability in certain realistic scenarios, but the choice of those scenarios has drastically impacted the algorithm I have developed.\\
%
%During development, I have always used realistic approximations of the capabilities of multirotor UAVs. Those can reach high accelerations but have relatively low maximum velocities compared to what fixed-wing aircraft can achieve. This makes those vehicles very agile, which is one of the contributing factors to their recent popularity. \\
%The assumption of this agility means that my algorithm cannot be applied to UAVs which do not have that property. One of the properties my algorithm uses often is the maximum acceleration distance, which is the distance in which the UAV can always come to a complete stop. This works fine with multirotor UAVs, but is a meaningless concept when dealing with fixed-wing UAVs which cannot stop at all during flight. \\
%However, those kinds of low-agility UAVs are unlikely to be deployed at low altitudes in dense city centers exactly because they lack agility. Even with perfect planning, cities are very unpredictable places. An multirotor UAV is much more likely to be able to safely react to an unexpected obstacle than a fixed-wing UAV. \\
%While I picked out fixed-wing UAVs as an example, the same arguments hold for any kind of UAV that either cannot hover or has low agility. Some UAVs may be able to hover, but are not very agile due to a high maximum velocity and low acceleration. In this case, the agility can be improved by reducing the maximum velocity.\\ \\
%
%The density of obstacles is also an extremely important factor. The Leuven scenario is significantly harder than the San Francisco scenario because the obstacles are smaller and closer together. Not only are the obstacles closer together, but they are also polygons and can have many more edges per obstacle. For scenarios where the obstacles are significantly denser or complex than in the Leuven scenario, the approach may not improve the performance enough. \\
%On the other hand, the Leuven map is much more detailed than is required for navigation. Many obstacles could be merged together without a significant on possible trajectories. The obstacles could also be simplified, reducing the amount of edges per obstacle. Given that the Leuven map is unprocessed except for calculating the convex hull of each obstacle, the algorithm should be able to handle most real world maps when properly prepared.
%
%Given these considerations, I conclude that my approach meets the basic requirements. The assumptions behind the design of the algorithm seem to be valid based on experiments.

\section{Important parameters}
\label{section:imp-params}
During development of the algorithm I settled on sensible default parameters which balance both the performance of the algorithm and the quality of the resulting trajectory. The experiments which tested different values for those parameters provide a deeper insight in the effects of those parameters. Many of those insights point to possible improvements to the current algorithm.

\subsection{Time Step Size and Maximum Time}
From the time step size experiment (section \ref{subsec:timestep}) and maximum time experiment (section \ref{subsec:maxtime}) it becomes clear that the amount of time steps in each segment has a very large effect on the performance of the algorithm. The time step size should be chosen so the quality of the trajectory is high enough to be usable, without being more detailed than necessary. How large the time step size should be will depend on the specific use case. The maximum time should always be as low as possible, while still ensuring that the goal can be reached in that time. \\
By combining the effects of those parameters, the algorithm could be made faster without suffering a quality penalty. For each  segment, the algorithm could first solve a MILP problem with a conservative maximum time and a large time step size. The solution to this problem shows how much time the UAV needed to reach its goal in that segment. This value can be used as a much tighter maximum time in a MILP problem with a smaller time step size. While I did not have time to properly implement this, a quick-and-dirty test showed promising results.  Using this method and solving the segments first with a time step size of $0.5s$, the San Francisco scenario MILP solve time dropped from 32s to 10s. For the Leuven scenario the MILP solve time dropped from  136s down to 38s. In both cases, the execution time was cut by more than two thirds without impacting the quality of the trajectory.
\paragraph{Divide and Conquer}
This fits in well with the "divide and conquer" approach used throughout the thesis. Not only is the trajectory planning problem solved as many smaller subtrajectories, but the algorithm also divides the kinds of problems that need to solved. The usage of Theta* separates the routing problem from the actual trajectory planning problem. The extension suggested here somewhat separates the optimization aspect from the trajectory planning problem. The tight time estimation provided by the coarse solution ensures that any viable trajectory in the fine MILP problem is necessarily also close to the optimal trajectory. Most of the optimization already happened in the coarse MILP problem.
\par
Further exploiting this divide and conquer approach will probably lead to even more improvements. Maybe it is possible to look at the slack variables of the edges of obstacles in the coarse solution and reduce the amount of edges which need to be modeled in the fine MILP problem? Maybe the motion of the UAV in the coarse trajectory can be used to place the transitions between segments in better locations? There are many possibilities which could be explored.

\subsection{Approach Margin}
The approach margin experiment (section \ref{subsec:approach-margin}) shows that having some approach margin is beneficial, but that the gains are often relatively small. Overlapping the segments by just 5 time steps already results in a significantly better trajectory than a very large approach margin. Because of this, I believe that my idea that a larger approach margin leads to more efficient approach is not accurate. I suspect that the slight improvements in trajectory score when increasing the approach margin are caused by simply having fewer segments. Or to put it more accurately: it is caused by having fewer transitions between segments. Overlapping the segments smooths out bad transitions between those segments. A larger approach margin does not improve bad transitions, it only guarantees that the UAV can correct for it in time. These bad transitions are not immediately obvious when the UAV is constantly maneuvering, but they become very clear when the UAV is flying straight. Figure \ref{fig:sf-wavy2} shows a case where the UAV is not moving entirely along the path after a turn. This is corrected, but it starts an oscillation along the trajectory that takes many segments to die down. In Figure \ref{fig:sf-wavy2b}, the slight overlap of the segments prevents that oscillation from starting in the first place. 
\begin{figure}[h]
	\centering
	
	\begin{subfigure}[t]{.4\textwidth}
        		\includegraphics[width=\textwidth]{img/sf-wavy2}
        		\caption{}
        		\label{fig:sf-wavy2}
	\end{subfigure}
	\hfill
	\begin{subfigure}[t]{.4\textwidth}
        		\includegraphics[width=\textwidth]{img/sf-wavy2b}
        		\caption{}
        		\label{fig:sf-wavy2b}
	\end{subfigure}	
	
        
    \caption[The effect of overlapping obstacles on oscillations in the trajectory]{Without overlapping the segments, oscillations tend to occur in the trajectory as seen in \ref{fig:sf-wavy2}. Even a small amount of overlap is an effective countermeasure as seen in \ref{fig:sf-wavy2b}.}
    \label{fig:sf-wavy}
\end{figure}
With my current algorithm there is a high performance cost involved with overlapping the segments. However, I do not believe this necessarily has to be the case. This is definitely something to look at in future work.

\subsection{Genetic Algorithm}
The genetic algorithm is the part of the thesis that got the least attention. This is mainly because the genetic algorithm itself is not an essential part of the trajectory planning algorithm. It is used to grow the convex safe region, and the current genetic algorithm does that well enough. Due to time constraints I have omitted detailed parameter tuning of this genetic algorithm.
\par
Either way, the genetic algorithm is very crude. A genetic algorithm was chosen to grow the convex safe region because it is a quick way to get a reasonably good result.  Since it never was the most pressing issue in the project, it never got replaced by a more refined algorithm (genetic or otherwise). 



\clearpage
%\input{nonconvex}
%\clearpage
\section{Conclusions}
\label{section:conclusions}
Path planning using MIP was previously not computationally possible in large and complex environments. The approach presented in this paper shows that these limitations can effectively be circumvented by dividing the path into smaller segments using several steps of preprocessing. The specific algorithms used in each step to generate the segments can be swapped out easily with variations. Because the final path is generated by a solver, the constraints on the path can also be easily changed to account for different use cases. The experimental results show that the algorithm works well in realistic, city-scale scenarios, even when obstacles are distributed irregularly and dense.

\subsection{Future work}
The results so far are promising, but have not been used on real hardware yet. Extending the software we built so it can be tested with actual hardware is an obvious next step. That also leads to the next possible extension: Currently the algorithm works in 2D, but extending it to 3D would allow it to be used in more kinds of environments. We'd also like to allow for more kinds of constraints on the path of the vehicle.
\clearpage

\appendixpage*
\appendix
\chapter{Paper}
\includepdf[pages=-]{dewaen-ecmr2017.pdf}
\chapter{Poster}
\includepdf{poster.pdf}

\backmatter

\bibliographystyle{plain}
\bibliography{../papers/bib.bib}
\end{document}