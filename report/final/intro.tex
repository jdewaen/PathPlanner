\chapter{Introduction}
As a consequence of ever-increasing automation in our daily lives, more and more machines have to interact with and unpredictable environment and other actors within that environment. One of the sectors that seems like it will change dramatically in the near future is the transportation industry. Autonomous cars are actually starting to appear on public roads, autonomous truck convoys are being tested and several large retail distributors are investing heavily into delivering orders with Unmanned Areal Vehicles (UAV) instead of by courier. 
\par
Multirotor UAVs, often called quadrocopters when they have four propellers, are rising quickly in popularity. This is due to a variety of factors. Multirotos UAVs are cheaper to build than helicopters and they are much more agile than airplanes. The continuous improvements in battery technology allow them to use electric motors, compared to the gasoline motors which were ubiquitous in model aircraft just a few years ago. Finally, advances in mobile computing and sensors (caused by the smartphone industry) allowed "fly-by-wire" technology as a standard. With fly-by-wire, a computer interprets the commands by a human controller while keeping the vehicle stabilized. This allows even laymen to fly a multirotor UAV with a minimal amount of practice even though multirotor UAVs are inherently unstable. This is in stark contrast with the complex interactions between forces and torques on helicopters which take extensive training to master.
\par
The combination of the rising availability of multirotor UAVs and the perpetually increasing automation spark the imagination of many. Media outlets often present a utopic (distopic?) future with swarms of UAVs delivering products to consumers only minutes after they have been ordered. In the industry there is also an increasing amount of interest in autonomous UAVs because they could carry out inspections on large structures which are hard to reach for humans.
\par
However, there are still many challenges that prevent the proliferation of autonomous UAVs. These include, legislative uncertainties, coordination of the increased air traffic, trajectory planning and more. This thesis focuses on the trajectory planning problem, and more specifically on trajectory planning in large and complex environments like cities.
\section{Problem Statement}
Typically, trajectory planning for UAVs is done "online" and in the short term. This means that the calculations happen in realtime while the UAV is flying and that the trajectory is only planned for the near future. This works well in open spaces with a limited amount of obstacles, but cannot be applied to complex environments like cities.
\par
The goal of this thesis is to develop a long-term trajectory planning algorithm for multirotor UAVs which can scale to large and complex environments like cities. The algorithm is developed for offline trajectory planning, which means that the trajectory is planned before the flight of the UAV. Long-term offline planners are complimentary to short-term online planners. Long-term plans ensure that a safe trajectory to the goal exists and can guide the short-term planner in the right direction.
\par
Mixed-Integer Linear Programming (MILP) is the most common technique used in short-term trajectory planners. It allows for a large amount of flexibility when modeling the trajectory planning problem. From this model, the optimal trajectory can be found by using widely-available solvers. However, performance challenges prevent this technique from being applied beyond short-term planning. This thesis aims to preserve the flexibility of MILP trajectory planning while making the technique scalable enough to solve planning problems on the scale of cities.

%One such challenge, which is especially important for UAVs, is trajectory planning. Even though most modern multirotor UAVs are capable of flying by themselves, they are unable to generate a flight trajectory that will get them to a destination reliably.
%While these developments look promising, there are still many challenges that prevent these systems from being widely deployed.
%
%Classic graph-based shortest path algorithms like Dijkstra's algorithm its many variants fail to take momentum and other factors into account. Mixed Integer Linear Programming (MILP) is one approach that shows promising results, however it is currently severely limited by computational complexity.
%


%\section{Motivation }
%One of the main advantages of using a constraint optimization approach like MILP is that they are extremely extendable by design. A system based on this can be deployed in many different scenarios with different goals and constraints without the need for significant changes to the algorithms that drive it. The solvers that construct the final path are general solvers which take constraints and a target function as input. This input can be generated by end users in the field to match their specific requirements, making the software controlling the drones as flexible as the hardware.
%\par
%That flexibility is also the main limitation of using constraint optimization. The solvers are general purpose, which make them very slow compared to more direct approaches. They need to be carefully guided solve all but the most basic scenarios in a reasonable amount of time. While there have been some good results on small scales, I could not find any attempts at planning paths on the order of kilometers or more. Practical use cases involving drones often involve several minutes of flight and can cover several kilometers, so a path planner must be able to work at such a scale. This is the main goal of this thesis: To demonstrate how a MILP approach can be scaled to scenarios with a much larger scope, while preserving the advantages that make it interesting.
\section{Contribution}
In the past, long-term plans for UAVs have usually been built by humans. Research into long-term trajectory planning for UAVs has been limited. I have not been able to find previous work which has focused entirely on the scalability aspect of MILP trajectory planning. This thesis presents an approach which can scale to large and complex environments. While there is plenty of room for improvement, this new approach is the main contribution of this thesis.
\par
The experiments performed to test the algorithm also point to some ways the approach can be adapted to improve the performance and quality of the generated trajectory. Because of the lack of earlier research into the scalability aspect of trajectory planning, this thesis contributes new insights into the challenges and opportunities associated with building a scalable MILP trajectory planning system.

%The goal for this thesis is to build an algorithm that uses Mixed-Integer Linear Programming for UAV trajectory planning. The algorithm must be scalable so it can handle long trajectories through large environments with many obstacles.
%\par

\section{Structure of the Thesis}
Section \ref{subsec:previous} summarizes the previous work that has been done in the field, presenting similar approaches as well as some  alternatives.
\par
The previous work in the field shows a common design to modeling the path planning problem as a MILP problem. This design forms the foundation of the MILP model used in this thesis. Chapter \ref{section:modelingbasic} details how this MILP model is constructed.
\par
Chapter \ref{section:segment} identifies the performance limitations which come along with this MILP approach. It also proposes an algorithm which preprocesses the trajectory planning problem to circumvent those performance limitations.
\par
Chapter \ref{section:extensions} presents several extensions to the algorithm to further improve the results. It also shows the graphical visualization tool which was developed to analyze the execution of the algorithm.
\par
Chapter \ref{section:analysis} presents a series of experiments which test the algorithm using a variety of environments and combinations between parameter values.
\par
Chapter \ref{section:discussion} discusses the results of the experiments and assesses whether or not the goal of this thesis has been reached. It also highlights some of the issues that are still present in the algorithm. Furthermore, it proposes several extensions and changes which could be added in the future to further improve the algorithm.
\par
Finally, Chapter \ref{section:conclusions} summarizes the main elements of this thesis and forms a final conclusion.

\section{Literature Review}
\label{subsec:previous}


Schouwenaars et al. \cite{Schouwenaars2001} were the first to demonstrate that MILP could be applied to path planning problems. They used discrete time steps to model time with a vehicle moving through 2D space, just like the approach we present in this paper. The basic formulations of constraints we present in this paper are the same as in the work of Schouwenaars et al. To limit the computation complexity, they also presented a receding horizon technique so the problem can be solved in multiple steps. However, this technique was essentially blind and could easily get stuck behind obstacles. Bellingham\cite{Bellingham2002} recognized that issue and proposed a method to prevent the path from get stuck behind obstacles, even when using a receding horizon. \\

Flores\cite{Flores2007} and Deits et al\cite{Deits2015} do not use discretized time, but model continuous curves instead. This not possible using linear functions alone. They use Mixed Integer Programming (MIP) with functions of a higher order to achieve this. The work by Deits et al. is especially relevant to this paper, since they also use convex regions to limit (or in their work: completely eliminate) the need to model obstacles directly. \\

Several papers \cite{Fliess1995a, Hao2005, Cowling2007, Mellinger2011} show how the output from algorithms like these can be translated to control input for an actual physical vehicle. This demonstrates that, when they properly model a vehicle, these path planners need minimal post-processing to control a vehicle. Of course that does assume these planners can run in real time to deal with errors that inevitably will grow over time. Culligan \cite{Culligan2006} provides an approach built with real time operation in mind. Their approach only finds a suitable path for the next few seconds of flight and updates that path constantly.TODO: explain better \\

More work has been done on modeling specific kinds of constraints or goal functions. For instance, Chaudhry et al. \cite{Chaudhry2004} formulated an approach to minimize radar visibility for drones in hostile airspace. However, none of these have really attempted to make navigating through a complex environment like a city feasible. The approach by Deits et al. \cite{Deits2015} could work, but did not really explore the effects of longer paths on their algorithm.

TODO:PRM and RERT \\

\section{Assumptions}
TODO