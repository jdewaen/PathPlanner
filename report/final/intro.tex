\section{Introduction}
As a consequence of ever-increasing automation in our daily lives, more and more machines have to interact with and unpredictable environment and other actors within that environment. One of the sectors that seems like it will change dramatically in the near future is the transportation industry. Autonomous cars are actually starting to appear on public roads, autonomous truck convoys are being tested and large retail distributors like Amazon are investing heavily into delivering order by drones instead of courier. While these developments look promising, there are still many challenges that prevent these systems from being widely deployed.
\par
One such challenge, which is especially important for areal vehicles, is path planning. Even though most modern quadrocopters are capable of flying by themselves, they are unable to generate a flight path that will get them to their destination reliably. Classic graph-based shortest path algorithms like Dijkstra's algorithm its many variants fail to take momentum and other factors into account. Mixed Integer Linear Programming (MILP) is one approach that shows promising results, however it is currently severely limited by computational complexity.



\subsection{Motivation }
One of the main advantages of using a constraint optimization approach like MILP is that they are extremely extendable by design. A system based on this can be deployed in many different scenarios with different goals and constraints without the need for significant changes to the algorithms that drive it. The solvers that construct the final path are general solvers which take constraints and a target function as input. This input can be generated by end users in the field to match their specific requirements, making the software controlling the drones as flexible as the hardware.
\par
That flexibility is also the main limitation of using constraint optimization. The solvers are general purpose, which make them very slow compared to more direct approaches. They need to be carefully guided solve all but the most basic scenarios in a reasonable amount of time. While there have been some good results on small scales, I could not find any attempts at planning paths on the order of kilometers or more. Practical use cases involving drones often involve several minutes of flight and can cover several kilometers, so a path planner must be able to work at such a scale. This is the main goal of this thesis: To demonstrate how a MILP approach can be scaled to scenarios with a much larger scope, while preserving the advantages that make it interesting.
\subsection{Contribution}
TODO...
\subsection{Structure of the Thesis}
Section \ref{subsec:previous} summarizes the previous work that has been done in the field.
The previous work in the field shows a common design to modeling the path planning problem as a MILP problem. This design forms the core of the approach in this thesis as well. Section \ref{section:modeling} shows the implementation of this common design and explores the critical limitations to this approach.
\par
Section \ref{section:segment} proposes a solution to these limitation. By finding a rough initial path, the planning problem can be split into smaller segments. Solving these segments on their own is significantly easier and can still enforce all the constraints. This approach is much faster than previous techniques, but at the cost of no longer finding the global optimum.
\par
During the development of this algorithm, finding and solving bugs and other unwanted behavior proved to be a significant challenge. A visualization tool was developed to make it easier to see how the algorithm operates. Section \ref{section:visual} goes into detail of how nearly every variable in the MILP problem was visualized and how this information can be interpreted.
\par
To demonstrate the flexibility of the approach, section \ref{section:extension} showcases some possible extensions that can be added with relative ease. Some of these have been fully implemented to look at the impact on the solution. This section should demonstrate that the path planner discussed in this thesis is a modular strategy built out of several different algorithms. The specific algorithms discussed are just one way of doing things, and can be easily swapped out for other, more advanced, algorithms. 
\par
Section \ref{section:result} analyzes the performance of the path planner in several different scenarios. It also looks at how the extensions which have been implemented affect the both the performance and quality of the planner.
Finally, section \ref{section:summary} summarizes the main observations in this thesis and concludes whether or not the goals have been realized.

\subsection{Assumptions}


\subsection{Literature Review}

TODO:PRM and RERT \\

Schouwenaars et al. \cite{Schouwenaars2001} were the first to demonstrate that MILP could be applied to path planning problems. They used discrete time steps to model time with a vehicle moving through 2D space, just like the approach we present in this paper. The basic formulations of constraints we present in this paper are the same as in the work of Schouwenaars et al. To limit the computation complexity, they also presented a receding horizon technique so the problem can be solved in multiple steps. However, this technique was essentially blind and could easily get stuck behind obstacles. Bellingham\cite{Bellingham2002} recognized that issue and proposed a method to prevent the path from get stuck behind obstacles, even when using a receding horizon. \\

Flores\cite{Flores2007} and Deits et al\cite{Deits2015} do not use discretized time, but model continuous curves instead. This not possible using linear functions alone. They use Mixed Integer Programming (MIP) with functions of a higher order to achieve this. The work by Deits et al. is especially relevant to this paper, since they also use convex regions to limit (or in their work: completely eliminate) the need to model obstacles directly. \\

Several different papers \cite{Fliess1995a, Hao2005, Cowling2007, Mellinger2011} show how the output from algorithms like these can be translated to control input for an actual physical vehicle. This demonstrates that, when they properly model a vehicle, these path planners need minimal post-processing to control a vehicle. Of course that does assume these planners can run in real time to deal with errors that inevitably will grow over time. Culligan \cite{Culligan2006} provides an approach built with real time operation in mind. Their approach only finds a suitable path for the next few seconds of flight and updates that path constantly. \\

More work has been done on modeling specific kinds of constraints or goal functions. For instance, Chaudhry et al. \cite{Chaudhry2004} formulated an approach to minimize radar visibility for drones in hostile airspace. However, none of these have really attempted to make navigating through a complex environment like a city feasible. The approach by Deits et al. \cite{Deits2015} could work, but did not really explore the effects of longer paths on their algorithm.
