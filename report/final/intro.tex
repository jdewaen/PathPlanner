\chapter{Introduction}
As a consequence of ever-increasing automation in our daily lives, more and more machines have to interact with and unpredictable environment and other actors within that environment. One of the sectors that seems like it will change dramatically in the near future is the transportation industry. Autonomous cars are actually starting to appear on public roads, autonomous truck convoys are being tested and several large retail distributors are investing heavily into delivering orders with Unmanned Areal Vehicles (UAV) instead of by courier. 
\par
Multirotor UAVs, often called quadrocopters when they have four propellers, are rising quickly in popularity. This is due to a variety of factors. Multirotos UAVs are cheaper to build than helicopters and they are much more agile than airplanes. The continuous improvements in battery technology allow them to use electric motors, compared to the gasoline motors which were ubiquitous in model aircraft just a few years ago. Finally, advances in mobile computing and sensors (caused by the smartphone industry) allowed "fly-by-wire" technology as a standard. With fly-by-wire, a computer interprets the commands by a human controller while keeping the vehicle stabilized. This allows even laymen to fly a multirotor UAV with a minimal amount of practice even though multirotor UAVs are inherently unstable. This is in stark contrast with the complex interactions between forces and torques on helicopters which take extensive training to master.
\par
The combination of the rising availability of multirotor UAVs and the perpetually increasing automation spark the imagination of many. Media outlets often present a utopic (distopic?) future with swarms of UAVs delivering products to consumers only minutes after they have been ordered. In the industry there is also an increasing amount of interest in autonomous UAVs because they could carry out inspections on large structures which are hard to reach for humans.
\par
However, there are still many challenges that prevent the proliferation of autonomous UAVs. These include, legislative uncertainties, coordination of the increased air traffic, trajectory planning and more. This thesis focuses on the trajectory planning problem, and more specifically on trajectory planning in large and complex environments like cities.
\section{Problem Statement}
Typically, trajectory planning for UAVs is done "online" and in the short term. This means that the calculations happen in realtime while the UAV is flying and that the trajectory is only planned for the near future. This works well in open spaces with a limited amount of obstacles, but cannot be applied to complex environments like cities.
\par
The goal of this thesis is to develop a long-term trajectory planning algorithm for multirotor UAVs which can scale to large and complex environments like cities. The algorithm is developed for offline trajectory planning, which means that the trajectory is planned before the flight of the UAV. Long-term offline planners are complimentary to short-term online planners. Long-term plans ensure that a safe trajectory to the goal exists and can guide the short-term planner in the right direction.
\par
Mixed-Integer Linear Programming (MILP) is the most common technique used in short-term trajectory planners. It allows for a large amount of flexibility when modeling the trajectory planning problem. From this model, the optimal trajectory can be found by using widely-available solvers. However, performance challenges prevent this technique from being applied beyond short-term planning. This thesis aims to preserve the flexibility of MILP trajectory planning while making the technique scalable enough to solve planning problems on the scale of cities.

\section{Contribution}
In the past, long-term plans for UAVs have usually been built by humans. Research into long-term trajectory planning for UAVs has been limited. I have not been able to find previous work which has focused entirely on the scalability aspect of MILP trajectory planning. This thesis presents an approach which can scale to large and complex environments. While there is plenty of room for improvement, this new approach is the main contribution of this thesis.
\par
The experiments performed to test the algorithm also point to some ways the approach can be adapted to improve the performance and quality of the generated trajectory. Because of the lack of earlier research into the scalability aspect of trajectory planning, this thesis contributes new insights into the challenges and opportunities associated with building a scalable MILP trajectory planning system.
\par
All code and data used in this thesis are available on \url{https://github.com/jdewaen/PathPlanner}.

%The goal for this thesis is to build an algorithm that uses Mixed-Integer Linear Programming for UAV trajectory planning. The algorithm must be scalable so it can handle long trajectories through large environments with many obstacles.
%\par

\section{Structure of the Thesis}
Section \ref{subsec:previous} summarizes the previous work that has been done in the field, presenting similar approaches as well as some  alternatives.
\par
The previous work in the field shows a common design to modeling the path planning problem as a MILP problem. This design forms the foundation of the MILP model used in this thesis. Chapter \ref{section:modelingbasic} details how this MILP model is constructed.
\par
Chapter \ref{section:segment} identifies the performance limitations which come along with this MILP approach. It also proposes an algorithm which preprocesses the trajectory planning problem to circumvent those performance limitations.
\par
Chapter \ref{section:extensions} presents several extensions to the algorithm to further improve the results. It also shows the graphical visualization tool which was developed to analyze the execution of the algorithm.
\par
Chapter \ref{section:analysis} presents a series of experiments which test the algorithm using a variety of environments and combinations between parameter values.
\par
Chapter \ref{section:discussion} discusses the results of the experiments and assesses whether or not the goal of this thesis has been reached. It also highlights some of the issues that are still present in the algorithm. Furthermore, it proposes several extensions and changes which could be added in the future to further improve the algorithm.
\par
Finally, Chapter \ref{section:conclusions} summarizes the main elements of this thesis and forms a final conclusion.

\section{Literature Review}
\label{subsec:previous}
Schouwenaars et al.\cite{Schouwenaars2001} were the first to demonstrate that MILP could be applied to trajectory planning problems. They used discrete time steps to model time with a vehicle moving through 2D space. Obstacles are modeled as grid-aligned rectangles. To limit the computational complexity, they presented a receding horizon technique so the problem can be solved in multiple steps. However, this technique is essentially blind and could easily get stuck behind obstacles. Bellingham\cite{Bellingham2002} recognized that issue and proposed a method to prevent the trajectory from getting stuck behind obstacles, even when using a receding horizon. However Bellingham's approach still scales poorly in environments with many obstacles. The basic formulations of constraints of the MILP model in this thesis are extensions of the work by Bellingham.
\par
Flores\cite{Flores2007} and Deits et al.\cite{Deits2015} do not use discretized time, but model continuous curves instead. This not possible using linear functions alone. They use Mixed Integer Programming with functions of a higher order to achieve this. 
\par
Several papers \cite{Fliess1995a, Hao2005, Cowling2007, Mellinger2011} show how the four motors on a quadrocoptor can be controlled based on 3 positional coordinates and a single angular coordinate (and their derivatives) of the UAV. This demonstrates that even trajectories in which the motors are not modeled can be used to control a UAV. However, since there will always be small perturbations, an online trajectory planner is needed to keep the UAV on track. Culligan \cite{Culligan2006} provides an approach built with real time operation in mind. Their approach finds a suitable path for the next few seconds of flight and updates that path constantly. Others \cite{Kamal2005} \cite{Luders2008} have presented similar approaches.
\par
More work has been done on modeling specific kinds of constraints or goal functions, with a notable presence of military applications \cite{Maillot2015}\cite{Chaudhuri2015}. For instance, Chaudhry et al. \cite{Chaudhry2004} formulated an approach to minimize radar visibility for drones in hostile airspace. UAVs have been  commonly deployed in military context for much longer than for civilian purposes, so the presence of this research is not unexpected.
\par
Mathematical optimization approaches, using techniques like MILP, are the most common to solve the trajectory planning problem. The advantage of mathematical optimization approaches is that they can model different kinds of constraints without having to plan ahead for this while designing the algorithm. There are several papers \cite{Jung2008}\cite{Askari2016}\cite{Gao2013}\cite{DeFilippis2012} which use path planning algorithms which are modified to generate smooth paths which can be follow by UAVs. However, extending these algorithms to take other constraints into account often is very difficult or simply not possible.
\par
Finally there are the randomized approaches using probabilistic roadmaps (PRM)\cite{Naderi2015} \cite{Kavraki1996}\cite{Saha2006} and rapidly-exploring random trees (RRT) \cite{LaValle1998}\cite{Tsai2015}\cite{Naderi2015}. These algorithms can be very fast compared to the optimization approaches, and RRT can even take arbitrary constraints into account. However their randomized nature leads to artifacts in the path. Furthermore, even though RRT can model constraints, the many degrees of freedom  (position, velocity, acceleration, and possibly more derivatives) also present scalability issues for this approach when planning long trajectories.
\par
Many of the papers I have cited focus on navigation through airspace without obstacles. Some take collision avoidance with other aircraft into account. Of the few approaches which use MILP (or similar optimization techniques) and take obstacle avoidance into account, only the approach by Deits et al. \cite{Deits2015} considers the performance issues caused by those obstacles. They present an approach which bears some similarities with the one presented in this thesis by using convex safe spaces. This allows their approach to scale well in environments with many obstacles. However they scalability with regards to much longer trajectories.

\section{Assumptions}
Trajectory planning is a complex subject with many facets. Several assumptions were made to reduce the scope to a realistic size for a thesis.
\begin{itemize}
\item The UAV travels through 2D space and does not have an orientation. This assumption limits the complexity of the trajectory planning problem so the focus can be on the scalability aspect 
\item All obstacles are known in advance and static.
\item The UAV is the only vehicle in the world.
\item The UAV can come to a stop during flight. This is a property shared by multirotor vehicles as well as helicopters, but not by airplanes.
\end{itemize}