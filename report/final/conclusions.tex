\section{Conclusions}
\label{section:conclusions}
Path planning using MIP was previously not computationally possible in large and complex environments. The approach presented in this paper shows that these limitations can effectively be circumvented by dividing the path into smaller segments using several steps of preprocessing. The specific algorithms used in each step to generate the segments can be swapped out easily with variations. Because the final path is generated by a solver, the constraints on the path can also be easily changed to account for different use cases. The experimental results show that the algorithm works well in realistic, city-scale scenarios, even when obstacles are distributed irregularly and dense.

\subsection{Future work}
The results so far are promising, but have not been used on real hardware yet. Extending the software we built so it can be tested with actual hardware is an obvious next step. That also leads to the next possible extension: Currently the algorithm works in 2D, but extending it to 3D would allow it to be used in more kinds of environments. We'd also like to allow for more kinds of constraints on the path of the vehicle.