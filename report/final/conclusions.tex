

\chapter{Conclusions}
\label{section:conclusions}

\section{Reflection}
The goal for this thesis was relatively open-ended: to build a scalable offline trajectory planning algorithm for multirotor UAVs. This scalability entailed that the algorithm had to work in very large, complex and realistic scenarios. The experiments show that the algorithm is capable of planning trajectories through such environments, and that it can do so consistently. There are still some minor issues with the algorithm, but the results show that clearly this approach is viable.
\par
Even though the goals of this thesis have certainly been reached, the algorithm presented in this thesis is just a first step. The algorithm shows that MILP trajectory planning can scale to environments which are several orders of magnitude more complex than what has been considered before. However, the algorithm still has weaknesses, especially around segment transitions. These transitions occasionally fail, and can be less efficient than desired.
\par
One of the main challenges of this thesis was the lack of prior research into this subject. While there is plenty of material available on how to build a MILP trajectory planning model, the performance characteristics of those models are mostly unexplored. This made it difficult to identify which part of the problem to focus on next. With the insights from this thesis, I believe that it is possible to improve upon this algorithm and reach even better results.
\par
I believe that the main contribution of this thesis to the field is   not be the specific algorithm I have developed. The experimental results show that solving different aspects of the trajectory planning problem separately allows for large performance improvements. This principle could be applied in other ways than the one presented in this thesis.
\newpage
\section{Future work}
\label{section:future}
There are several extensions and improvements which may improve the performance of the algorithm or the quality of the trajectory. These extensions may also improve the general utility of the algorithm, making it useful for different use cases.

\begin{itemize}
\item As discussed in section \ref{section:imp-params}, solving each segment multiple times with different parameters can result in performance improvements. However, when combined with the observation that the segment transition can be improved with overlapping segments, I believe that a more drastic change may be in order. Solving the segments first with a coarse time step size may also be used in combination with a high amount of overlap with other segments. This rough trajectory does not only contain information on the time needed to solve a segment, but may also be used to divide the problem into segments differently.
\item Another step is extending this approach to 3D. The extra degree of freedom will likely come at a significant performance penalty, so this was not attempted during the thesis. One of the likely difficulties with the preprocessing as presented is that it treats all dimensions the same. This is fine for the horizontal dimensions, but due to gravity, movements the vertical dimension have different characteristics. The maximum acceleration of the UAV can no longer be assumed to be the same in all directions.\\
A possible mitigation to the increasing complexity of obstacles may be using a "2.5D" representation. A 2.5D obstacle is a 2D obstacle which also has a height value. This would only need one additional integer variable per obstacle to model. In a city scenario, this may be an acceptable approximation.
\item I would like to try using Mixed-Integer Quadratic programming. This eliminates the need to approximate the 2-norm. This may improve performance.
\item The algorithm presented in this thesis is aimed at offline planning. To test the algorithm on actual physical UAVs, an online planner is necessary. This online planner will need to be able to use the trajectory generated by this algorithm and ensure that the UAV keeps following it even if there are perturbations.
\end{itemize}
\newpage
\section{Summary}
In this thesis, I presented a scalable trajectory planning algorithm using Mixed-Integer Linear Programming (MILP). This algorithm is capable of generating trajectories through environments on the scale of cities in 2D space. These environments can be several square kilometers in size with trajectories spanning several kilometers. Previous approaches with MILP trajectory planning were not scalable enough to generate trajectories through such environments.
\par
This performance improvement is achieved by using several steps of preprocessing. This preprocessing approach is the main contribution of this thesis to the field. During preprocessing, a Theta* path planning algorithm is used to find a viable route to reach the goal. Based on this path, the trajectory planning problem is divided into many subproblems or segments, which each solve a small part of the final trajectory. The segments are solved consecutively and their results are stitched together to form the final trajectory.
\par
Dividing the trajectory planning problem reduces the amount of time steps and obstacles that have to be modeled in each subproblem. Minimizing both those properties is critical to ensure the MILP subproblem can be solved quickly. Special care was taken to ensure that the transitions between the segments do not cause the algorithm to fail.
\par
The algorithm was tested with a variety of scenarios. Several of those scenarios are based on maps of actual cities, namely San Francisco and Leuven. The results show that the MILP part of the algorithm scales linearly (instead of exponentially) with the trajectory length and is dependent on the density of the obstacles instead of the total amount of obstacles. The Theta* algorithm still scales exponentially, but is a much easier problem to solve with a large body of research detailing possible improvements.
\par
The transitions between segments present some challenges which have yet to be solved. They cause some inefficient movements which reduce the quality of the trajectory unnecessarily. Another weak point of the algorithm is the limited processing on the obstacles. Some preprocessing to simplify dense arrangements of obstacles would allow the algorithm to operate in even more complex environments.
\par
Detailed analysis of the effects of some of the parameters have shown that there is still a lot of room for improvement for the algorithm. Further development of the algorithm based on the insights gained through this thesis can address some of the shortcomings of the current approach and continue to exploit the strengths.

%Path planning using MIP was previously not computationally possible in large and complex environments. The approach presented in this paper shows that these limitations can effectively be circumvented by dividing the path into smaller segments using several steps of preprocessing. The specific algorithms used in each step to generate the segments can be swapped out easily with variations. Because the final path is generated by a solver, the constraints on the path can also be easily changed to account for different use cases. The experimental results show that the algorithm works well in realistic, city-scale scenarios, even when obstacles are distributed irregularly and dense.

