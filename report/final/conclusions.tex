

\chapter{Conclusions}
\label{section:conclusions}
In this thesis, I presented a scalable trajectory planning algorithm using Mixed-Integer Linear Programming (MILP). This algorithm is capable of generating trajectories through environments on the scale of cities in 2D space. These environments can be on the order of square kilometers in size with trajectories spanning several kilometers. Previous approaches with MILP trajectory planning were not scalable enough to generate trajectories through such environments.
\par
This improvement of performance is achieved by using several steps of preprocessing. This preprocessing approach is the main contribution of this thesis to the field. During preprocessing, a Theta* path planning algorithm is used to find a viable route to reach the goal. Based on this path, the trajectory planning problem is divided into many subproblems or segments, each of which solve a small part of the final trajectory. The segments are solved consecutively and their results are stitched together to form the final trajectory.
\par
Dividing the trajectory planning problem reduces the amount of time steps and obstacles that have to be modeled in each subproblem. Minimizing both those properties is critical to ensure the MILP subproblem can be solved quickly. Special care was taken to ensure that the transitions between the segments do not cause the algorithm to fail.
\par
The algorithm was tested with a variety of scenarios. Several of those scenarios are based on maps of actual cities, namely San Francisco and Leuven. The results show that the MILP part of the algorithm scales linearly (instead of exponentially) with the trajectory length and is dependent on the density of the obstacles instead of the total amount of obstacles. The Theta* algorithm still scales exponentially, but is a much easier problem to solve with a large body of research detailing possible improvements.
\par
Detailed analysis of the effects of some of the parameters have shown that there is still a lot of room for improvement for the algorithm. 

%Path planning using MIP was previously not computationally possible in large and complex environments. The approach presented in this paper shows that these limitations can effectively be circumvented by dividing the path into smaller segments using several steps of preprocessing. The specific algorithms used in each step to generate the segments can be swapped out easily with variations. Because the final path is generated by a solver, the constraints on the path can also be easily changed to account for different use cases. The experimental results show that the algorithm works well in realistic, city-scale scenarios, even when obstacles are distributed irregularly and dense.

