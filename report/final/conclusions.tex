

\chapter{Conclusions}
\label{section:conclusions}
Path planning using MIP was previously not computationally possible in large and complex environments. The approach presented in this paper shows that these limitations can effectively be circumvented by dividing the path into smaller segments using several steps of preprocessing. The specific algorithms used in each step to generate the segments can be swapped out easily with variations. Because the final path is generated by a solver, the constraints on the path can also be easily changed to account for different use cases. The experimental results show that the algorithm works well in realistic, city-scale scenarios, even when obstacles are distributed irregularly and dense.
\section{Future work}


%Another point that warrants attention are the transition between segments. In some cases, the UAV may end a segment in a state which causes issues in the next segment. This may cause the next segment to not be solvable, or result in a strange and undesirable trajectory. Overlapping the segments does help, but the algorithm can still not guarantee that the next segment will be feasible. Backtracking does guarantee that the next segment will be feasible, but at a great computational cost.\\
%This is partly due to active region generated by the genetic algorithm. Often the results are good, but in some cases the active region is very restrictive. The algorithm itself could certainly be improved, but it will be hard to guarantee a good result. Forcing the region to be significantly large solves the issue, but that comes at computational cost that may be too large.\\
%
%However, I do believe that these transition issues can be solved. One of the next extensions I would try is solving the MILP problem first with a higher time step size. Using this to roughly solve the current and next segment to provide a suitable goal state (including velocity) for the first segment would ensure a good start for the next segment as well. This may also allow for a lower approach margin multiplier, because the proper approach is already determined. \\
%


The obvious next step is extending this approach to 3D. The extra degree of freedom will likely come at a significant performance penalty, so this was not attempted during the thesis. One of the likely difficulties with the preprocessing as presented is that it treats all dimensions the same. This is fine for the horizontal dimensions, but due to gravity, movements the vertical dimension have different characteristics. The maximum acceleration of the UAV can no longer be assumed to be the same in all directions.\\
A possible mitigation to the increasing complexity of obstacles may be using a "2.5D" representation. A 2.5D obstacle is a 2D obstacle which also has a height value. This would only need one additional integer variable per obstacle to model. In a city scenario, this may be an acceptable approximation. \\

Another extension I would try is using moving to Mixed Integer Quadratic programming. Linear approximations to limit the length of vectors works, but it also introduces artifacts into the path. Increasing the amount of constraints that model the norm helps minimize the impact, but comes at a performance cost. Stating those constraints directly as a quadratic function would reduce the amount of constraints needed per time step. Even though the performance cost of a more accurate linear approximation is limited, this could still improve performance while increasing the accuracy of the model. Especially when the problem is also extended to 3D, since the this would require the linear approximation of a sphere instead of a circle. \\

\section{actual contribution}
\section{goals reached?}


\section{challenges}