\begin{abstract}
Trajectory planning using Mixed Integer Linear Programming (MILP) is a powerful approach because vehicle dynamics and other constraints can be taken into account. However, it is currently severely limited by poor scalability.
This paper presents a new approach which improves the scalability regarding the amount of obstacles and the distance between the start and goal positions.
%
While previous approaches hit computational limits when the problem contains tens of obstacles, our approach can handle tens of thousands of polygonal obstacles successfully on a typical consumer computer. 
%
This performance is achieved by dividing the problem into many smaller MILP subproblems using two sets of heuristics. Each  subproblem models a small part of the trajectory. The subproblems are solved in sequence, gradually building the desired trajectory.
%
The first set of heuristics generate each subproblem in a way that minimizes its difficulty, while preserving stability. The second set of heuristics select a limited amount obstacles to be modeled in each subproblem, while preserving consistency.
%
To demonstrate that this approach can scale enough to be useful in real, complex environments, it has been tested on maps of two cities with trajectories spanning over several kilometers.
\end{abstract}

\section{Introduction}
Trajectory planning for multirotor UAVs is a complex problem because flying is inherently a dynamic process. Proper modeling of velocity and acceleration are required to generate a feasible trajectory that is both fast and safe, that is, the UAV should be able to effectively navigate corners while maintaining momentum. The fastest trajectory is not always the shortest one, since the UAV's velocity may be different. The UAV dynamics are often not the only constraints placed on the trajectory. Different laws in different countries also affect the properties of the trajectory. The operators of the UAV may also wish to either prevent certain scenarios or ensure that specific criteria are always met. \\
In this paper we present a scalable approach which is capable of generating fast and safe trajectories, while also being easily extensible by design. We model the trajectory planning problem as a Mixed Integer Linear Program (MILP). The trajectory is represented in discrete time steps where each step describes the UAV's dynamic state at that moment. 
% In MILP, the problem is modeled using mathematical equations as constraints. 
An objective function encodes one or more properties, like time or trajectory length, to be optimized. A general solver is then used to find the optimal solution for the problem. Because the problem is defined declaratively, additional constraints can easily be added.\\
We demonstrate our approach in 2D environments. We assume that all obstacles are polygons, static and known in advance. Our algorithm is designed for offline planning, ensuring that a feasible trajectory exists before the UAV starts executing its task. 
% These limitations were considered, but we decided to keep the problem simple and focus on demonstrating that the new approach does in fact work. 
% We believe that the approach should also be effective in 3D, although more work is necessary.\\
Other papers have used MILP for trajectory planning \cite{Schouwenaars2001}, their approaches could not be used to generate long trajectories through complex environments. Our main contribution is an approach which improves the scalability by dividing the problem into many MILP subproblems. Each subproblem models only a part of the trajectory. The subproblems are solved sequentially. A first set of heuristics uses a Theta* path to generate the subproblems. The second set of heuristics select which obstacles should be modeled in each subproblem, limiting the amount of obstacles that need to be modeled while ensuring that no collisions can occur.

%Our main contribution is a preprocessing pipeline which makes MILP trajectory planning more scalable. In our approach, we divide the trajectory into many smaller segments. Several steps of preprocessing collect information about the trajectory. This information is used to generate the smaller segments, as well as reduce the difficulty of each specific segment. \\
%The first step consists of finding an initial path using the Theta* algorithm. This trajectory does not take dynamic properties into account, making it faster to calculate. In the second step, the corners in this initial path are extracted and used to generate the segments. We define a corner to be a distinct change in the path's direction, with that change in direction being necessary because at least one obstacle is in-between the position where the turn starts and the goal position. The third step attempts to minimize the amount of obstacles that need to be modeled in each segment. A heuristic selects important obstacles for each segment. An obstacle is important if its absence could have a large impact on the trajectory. These obstacles are considered active and will be fully modeled in the MILP problem, ensuring that the vehicle will not collide with an active obstacle.\\ 
%To ensure that the trajectory does not intersect with the inactive obstacles, a safe area is constructed by a genetic algorithm. This safe area is constructed such that it does not contain inactive obstacles and is convex. The vehicle is constrained stay within the safe area at all times in the MILP problem. To avoid restricting the movement of the vehicle unnecessarily, the genetic algorithm attempts to maximize the size of the safe area.\\
Schouwenaars et al.\cite{Schouwenaars2001} were the first to demonstrate that MILP could be applied to trajectory planning problems. They used discrete time steps to model time with a vehicle moving through 2D space. Obstacles are modeled as grid-aligned rectangles. To limit the computational complexity, they presented a receding horizon technique so the problem can be solved in multiple steps. However, this technique is essentially blind and could easily get stuck behind obstacles. Bellingham\cite{Bellingham2002} recognized that issue and proposed a method to prevent the trajectory from getting stuck behind obstacles, even when using a receding horizon. However Bellingham's approach still scales poorly in environments with many obstacles. The basic formulations of constraints we present in this paper are extensions of the work by Bellingham.\\
Flores\cite{Flores2007} and Deits et al.\cite{Deits2015} do not use discretized time, but model continuous curves instead. This not possible using linear functions alone. They use Mixed Integer Programming with functions of a higher order to achieve this. The work by Deits et al. is especially relevant to this paper, since they also use convex safe regions to solve the scalability issues when faced with many obstacles.
%Several papers \cite{Fliess1995a, Hao2005, Cowling2007, Mellinger2011} show how the full state of a quadrocopter, including motor thrust, can be determined using only the 3D position of the vehicle's center of mass, the yaw and their derivatives. This demonstrates that, when the properties of a vehicle are accurately modeled, trajectory planners like the one in this paper need minimal post-processing to control a vehicle. Of course that does assume these planners can run in real time to deal with errors that inevitably will grow over time. Culligan \cite{Culligan2006} provides an online approach. However, their approach can only find a suitable trajectory for the next few seconds of flight and updates that trajectory constantly. \\
%More work has been done on modeling specific kinds of constraints or goal functions. For instance, Chaudhry et al. \cite{Chaudhry2004} formulated an approach to minimize radar visibility for drones in hostile airspace. However, none of these have really attempted to make navigating through a complex environment like a city feasible. The approach by Deits et al. \cite{Deits2015} could work, but did not explore the effects of longer trajectories on their algorithm.