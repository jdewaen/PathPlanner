\section{Introduction}
Path planning is currently an unsolved problem for unmanned drones. Even though most modern quadrocopters are capable of flying by themselves, they are unable to generate a flight path that will get them to their destination reliably. Classic graph-based shortest path algorithms like Dijkstra's algorithm its many variants fail to take momentum and other factors into account. Mixed Integer Linear Programming (MILP) is one approach that shows promising results, however it is currently severely limited by computational complexity.


\subsection{Motivation}
One of the main advantages of using a constraint optimization approach like MILP is that they are extremely extendable by design. A system based on this can be deployed in many different scenarios with different goals and constraints without the need for significant changes to the algorithms that drive it. The solvers that construct the final path are general solvers which take constraints and a target function as input. This input can be generated by end users in the field to match their specific requirements, making the software controlling the drones as flexible as the hardware.
\par
That flexibility is also the main limitation of using constraint optimization. The solvers are general purpose, which make them very slow compared to more direct approaches. They need to be carefully guided solve all but the most basic scenarios in a reasonable amount of time. While there have been some good results on small scales, I could not find any attempts at planning paths on the order of kilometers or more. Practical use cases involving drones often involve several minutes of flight and can cover several kilometers, so a path planner must be able to work at such a scale. This is the main goal of the thesis: To demonstrate how a MILP approach can be scaled to scenarios with a much larger scope, while preserving the advantages that make it interesting.